%% conclusion.tex
%%

%% ==================
\chapter{Conclusion and Outlook}
\label{ch:conclusion}
%% ==================
This thesis introduced the concept of long-haul truck driver routing as the problem of finding shortest routes from a start to a target that require travel times up to multiple days and comply with regulatory constraints for driving times, mandatory breaks, and rest periods. We discussed real-life examples of the regulations of the EU and the US and abstracted from them to obtain the Truck Driver Routing Problem. The TDRP models regulations using a set of driving time constraints, each driving time consisting of a maximum allowed driving time and a minimum mandatory break time. In our main work in Chapter~\ref{ch:Algorithm}, we presented a label-based extension of Dijkstra's algorithm to solve the TDRP with one or two driving time constraints. We then proceeded to introduce a goal-directed algorithm that uses a potential that is based on CH-Potentials to obtain a tight lower bound for the travel time of a label to the target. Finally, an algorithm was introduced which combines the goal-directed approach with a core contraction hierarchy. For bidirectional variants of the algorithm, such as the core CH variants, we also described an additional pruning method as a further optimization.

In Chapter~\ref{ch:Evaluation}, we presented a rich series of experiments to evaluate the performance and behavior of our algorithms. Our best algorithm, which is the goal-directed core CH algorithm, shows running times of random queries on a European road network of \SI{339}{\milli\second} on average with a median of \SI{43}{\milli\second}. Furthermore, the goal-directed core CH algorithm never exceeded a running time of about one second in our experiments, even if considering individual outliers of the running time. In more experiments, we showed that the running time of the algorithm is influenced by different choices for the parameters of the driving time constraints and the set of parking nodes $P$. Some choices for $P$ can lead to high node degrees in the core CH which leads to long preprocessing times or even prohibits the contraction of all non-parking nodes, thus leading to a larger core. We showed that the running times of the queries begin to increase slowly when increasing the size of the core and a larger core thus can be used to decrease the preprocessing times significantly without loosing much of the queries' performance.

An interesting future task, especially for practical applications of the algorithm, is to incorporate an exact model of the EU driver's working hours or US hours of service regulations into our algorithm. Examples of such models exist for (including, but not limited to) the EU \cite{goel:2009a} and the US \cite{goel:2012c}. It also remains to extend the goal-directed algorithm and its proof of correctness for an arbitrary number of driving time constraints (TDRP-mDTC). In practice, no reasonable model of real-world regulations requires an unrestricted number of driving time constraints. The TDRP-mDTC most importantly raises an interesting theoretical question: How does the asymptotic complexity increase depending on the number of driving time constraints? We have shown that the running times increase for the TDRP-2DTC in comparison to the TDRP-1DTC. It remains to analyze the influence of additional driving time constraints for large values.


