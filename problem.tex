\chapter{Problem Definition}\label{ch:problem_definitions}
In our introduction in Chapter~\ref{ch:introduction}, we introduced the Long-Haul Truck Driver Routing Problem as a problem which arises from the practical challenge of finding optimal multi-day routes without violating regulations regarding truck drivers' driving times, working hours, breaks, and rest periods. In this chapter, we provide a formal definition of this problem using abstractions from the regulations which we described in the Sections~\ref{sec:dwh_eu}~and~\ref{sec:hos_us}.

Both, the drivers' working hours regulations of the EU and the hours of service regulations of the US are characterized by repeating cycles, consisting of a limited time period in which the driver is actively driving, and a period in which the driver must rest or may only conduct other work for a minimum amount of time. We model this characteristic using a set of \emph{driving time constraints}. A driving time constraint $\restr$ is a pair of two values $(c^d,c^b)$, a maximum allowed driving time or driving time limit $\restr^d$ and a minimum break time $\restr^b$. A driver must take an uninterrupted break with a length of at least $\restr^b$ before exceeding an accumulated driving time of $\restr^d$ since the last break with also a length of at least $\restr^b$. A set of multiple driving time constraints is denoted as the set $\restrset$ of driving time constraints $\restr_i \in \restrset$. We assume an order among the constraints $\restr_i$, a driving time constraint with a given index consists of a driving time limit and a break time which are each greater than or equal to a driving time constraint with a smaller index.

We now formalize our routing problem as an extension of the shortest path problem which accounts for driving time limits and mandatory breaks, which we name the Truck Driver Routing Problem (TDRP). Let $G=(V,E,\mathfunction{len})$ be a graph and $s$ and $t$ nodes with $s,t \in V$. We extend the graph $G$ to obtain a graph $G'=(V,P,E,\len)$ with a set $P \subseteq V$ of parking nodes.

\begin{definition}[Break Time]
	Given a path $p = \langle s=v_0,v_1,...,t=v_k \rangle$, the function $\breakTime\colon p \rightarrow \mathbb{R}_{\ge 0}$ assigns each node $v_i$ a non-negative break time $\breakTime(v_i)$.
\end{definition}

\begin{definition}[Route]
	A route $\route$ from $s$ to $t$ is defined by the path of visited nodes $p = \langle s=v_0,v_1,\dots,t=v_k \rangle$ and a break time function $\breakTime\colon p \rightarrow \mathbb{R}_{\ge 0}$.
\end{definition}

We extend the definition of the break time function \breakTime\ to obtain the break time of an entire route $\route$ using a path $p$ as $\breakTime(\route) = \sum_{v \in p}{\breakTime(v)}$.

\begin{samepage}
	\begin{definition}[Feasible Route]
		Let $G=(V,P,E,\len)$ be a graph with parking nodes and $\route$ a route in the graph using a path $p = \langle s=v_0,v_1,...,t=v_k \rangle$ and a break time function $\breakTime$. Let $\restrset$ be a set of driving time constraints $\restr_i \in \restrset$. The route $\route$ is feasible given $C$ if $\breakTime(v_i) = 0$ $\forall v_i \notin P$ and if there is no subpath between any consecutive nodes with a break time of at least $\restr_i^b$ which exceeds the maximum allowed driving time $c_i^d$.
	\end{definition}
\end{samepage}

We define the travel time of a route and the shortest route between two nodes as follows.

\begin{definition}[Travel Time of a Route]
	The travel time $\concretett(\route)$ of a route $\route$ is the sum of the length of its path $\len(p)$ and the accumulated break time $\breakTime(r)$.
\end{definition}

\begin{definition}[Shortest Route]
	A route between two nodes $s$ and $t$ is called a shortest route if it is feasible and there exists no other feasible route between $s$ and $t$ with a smaller travel time.
\end{definition}

The shortest travel time between two nodes $s$ and $t$, i.e., the travel time of the shortest route between them is denoted as $\traveltime(s,t)$. The TDRP can now be defined as follows.

\begin{namedproblem}
	\problemtitle{\textsc{Truck Driver Routing Problem}}
	\probleminput{A graph with parking nodes $G=(V, P, E,\len)$, a set of driving time constraints $\restrset$, start and target nodes $s,t \in V$}
	\problemquestion{Find a shortest route $r$ from $s$ to $t$ in $G$.}
\end{namedproblem}

We differentiate the cases in which we allow an arbitrary number of driving time constraints (TDRP-mDTC) or restrict the number of constraints to a certain number. We model the regulations of the EU and the US using a set $\restrset$ of two driving time constraints.

The EU's regulations are designed using the two central concepts of a break of \SI{45}{\minute} after a maximum driving time of \SI{4.5}{\hour} and a rest time of \SI{11}{\hour} after a maximum driving time of \SI{9}{\hour}. We therefore model the EU's regulations as $\restrset_{EU} = \{\restr_1, \restr_2\}$ with $\restr_1^d = \SI{4.5}{\hour}$, $\restr_1^b = \SI{0.75}{\hour}$, $\restr_2^d = \SI{9}{\hour}$, and $ \restr_2^b = \SI{11}{\hour}$.

The US regulations are centered around the concept of a break of \SI{30}{\minute} after at most \SI{8}{\hour} of driving and a mandatory off-duty time of \SI{10}{\hour} after \SI{11}{\hour} of driving. We ignore the \SI{14}{\hour} limit of on-duty time because the driver is not allowed to drive during the additional \SI{3}{\hour}, rendering the rule uninteresting for our routing problem. This leads to the set of driving time constraints $\restrset_{US} = \{\restr_1, \restr_2\}$ with $\restr_1^d = \SI{8}{\hour}$, $\restr_1^b = \SI{0.5}{\hour}$, $\restr_2^d = \SI{11}{\hour}$, and $\restr_2^b = \SI{10}{\hour}$.

As demonstrated above, we can model the most important characteristic of real-world driving time regulations using two constraints (TDRP-2DTC). In the regulatory framework of the EU and the US, this allows travel times up to a week with a net driving time of about \SI{60}{\hour}. We do not consider the TDRP with a model which uses one driving constraint (TDRP-1DTC) or with any restriction of the number of breaks on a route a Long-Haul Truck Driver Routing Problem since it does not allow multi-day routes with a realistic parameter setting for driving time limits and break times.