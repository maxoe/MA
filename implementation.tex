% !TeX root = thesis.tex
%% implementation.tex
%%

%% ==============
\chapter{Improvements and Implementation\label{ch:impl}}
In this chapter, we describe detailed modifications to the algorithm described in section \ref{sec:astar_corech} which enable further improvements of running time in specific cases and in practice.

\subparagraph{Label Queue}
In section \ref{sec:dijkstra_csp}, we introduced our basic approach as an algorithm which maintains one queue $Q$ of labels in contrast to Dijkstra's algorithm, which maintains one queue of nodes. In practice, we revert this change and $Q$ becomes a queue of nodes again. The key of a node $v$ in $Q$ is the best known travel time of a label in $L(v)$. The label set $L(v)$ now becomes a priority queue of labels itself. When settling a label, we remove a node $v$ from $Q$ and the best label $l$ at $v$ from $L(v)$. If $L(v)$ is not empty now, we insert the node $v$ into $Q$ again with the travel time of the new best label $l' \in L(v)$ as the key. TODO WHAT IF PROPAGATING A LABEL IS NEW BEST?

\subparagraph{Bidirectional Backward Pruning}
With a bidirectional A* search, we can use the progress and the potential of the backward search to prune the forward search and vice versa. [TODO cite] A label $l$ at a node $v$ which was propagated along an edge $(u,v)$ can be discarded if we can proof that all paths using the label are longer or equal to the tentative travel time $\tenttraveltime(s,t)$. We know the travel time $\concretett(l)$ of the label $l$ at $v$ and need to find a lower bound for the remaining distance to $t$. We will describe the pruning of the forward search, the backward search can be pruned accordingly.

The backwards queue $\overleftarrow{Q}$ contains labels with a key of $\concretett(l) + \concretepotential_s(l,v)$ for label $l \in \overleftarrow{L}(v)$. Labels are removed from the queue with an increasing key. Therefore, we know that if a label  TODO continue

\subparagraph{Core Contraction Hierarchy Stopping Criteria}
TODO when one search has no non core nodes left and the other search didn't reach core
\subparagraph{Constructing the Core Contraction Hierarchy}
