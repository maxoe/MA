% !TeX root = thesis.tex
%% related_work.tex
%%

%% ==============
\chapter{Related Work\label{ch:related_work}}
%% ==============
The planning of breaks on a route in a routing or scheduling context has been studied beforehand. The introduction of regulation EC 561/2006 of the European Union \cite{europeanparliament:2006} which regulates driving and rest times has led to the work of \cite{goel:2009}. The authors introduce a model for the regulation in the context of the Truck Driver Scheduling Problem (TDSP) which they solve using a label propagation algorithm. They revisit the TDSP in later publications, addressing regulations of different countries, such as the US \cite{goel:2012}. The authors of \cite{sartori:20210107} revisit the TDSP and name the respective variants EU-TDSP for the EU and US-TDSP for the US. All variants have in common that breaks must be taken on a route in regular intervals. The master's thesis of \cite{bomsdorf:2020} introduces the Truck Driver Scheduling and Routing Problem (TDSRP) as a generalization of the TDSP where the driver may also take breaks at specified locations, i.e. parking areas, between customers. They restrict the number of types of driving time constraints to two (TDSRP-2B) and provide an exact solution using mixed integer linear programming and additionally a heuristic approach. They also extend their work to solve a time-dependent variant of the problem (TD-TDSRP-2B). The TDSRP-2B in contrast to the TDSP is already related to the long-haul truck driver routing problem (LHTDRP). The TDSRP-2B contains the routing with regard for parking areas and driving time constraints between two customers, which resembles the LHTDRP.

The LHTDRP itself or variations of it were also addressed in previous work. The work of \cite{kleff:2017} restricts the LHTDRP by only allowing one break on a route, but includes traffic predictions using time-dependent road networks. It uses a version of core contraction hierarchies where all the parking nodes are part of the core, which we will revisit later in this work. The authors of \cite{tuin:2018} also allow only one break on a route and combine their approach with temporary driving bans and road closures. First, they introduce a multi-label version of Dijkstra's algorithm. Second, they present a heuristic version using time-dependent CH with an improved runtime in comparison.  In \cite{kleff:2020}, the long-haul truck driver routing problem is restricted to one type of driving time constraint but with an unlimited amount of breaks on a route. The work also addresses temporary driving bans, road closures, and a rating of parking areas. It therefore searches for pareto-optimal solutions with regard to travel time and ratings of the used parking areas. The approach is too slow to be used in practice, but it uses a combination of the A* algorithm and CH-Potentials which we also will revisit later.

Other publications address different complex real-world scenarios which present an extension to the SPP. As in this work, most of them strive to integrate the additional requirements into CH or other speed-up techniques which exploit the hierarchy of road networks \cite{bast:2015}. Integrating traffic predictions, respectively time-dependent road networks with CH is addressed by \cite{batz:2009,batz:2013}. Turn costs are addressed by \cite{geisberger:2011}. Mitigating the complex task of integrating additional information into CH is addressed by \cite{strasser:2021} by building a tight potential based on CH which then is used in combination with the A* algorithm. They use the same principle to address live traffic and extend the CH variant to a customizable contraction hierarchy \cite{dibbelt:2015} to address temporary driving bans.

TODO: \cite{mayerle:2020} - neccessary?