% !TeX root = thesis.tex
%% related_work.tex
%%

%% ==============
\chapter{Related Work\label{ch:related_work}}
%% ==============
The planning of breaks on a route in a routing or scheduling context has been studied beforehand. The introduction of regulation EC 561/2006 of the European Union \cite{europeanparliament:2006} which regulates driving and rest times has led to the work of \cite{goel:2009} who introduce a model for the regulation in the context of the Truck Driver Scheduling Problem (TDSP) which they solve using a label propagation algorithm. In the TDSP, a schedule must be found to visit multiple customers with given distances between them. Each customer must be visited in a given time window. The authors revisit the TDSP in later publications, addressing regulations of different countries, such as the US \cite{goel:2012}. The authors of \cite{sartori:20210107} revisit the TDSP and name the respective variants EU-TDSP for the EU and US-TDSP for the US. All variants have in common that breaks must be taken on a route in regular intervals. In \cite{goel:2012a}, the authors introduce the Minimum Truck Driver Scheduling Problem (MD-TDSP) which aims to find a valid truck driver schedule with a minimal total duration while breaks can only be taken at customers and specified rest areas. The problem is solved using mixed integer linear programming. The Truck Driver Scheduling and Routing Problem (TDSRP) presents an extension of the TDSP and has been studied with slightly varying definitions. The work of \cite{shah:2008} solves a time-dependent version the TDSRP heuristically and allows breaks on every node, there are no dedicated parking nodes. The Bachelor's thesis of \cite{braeuer:2016} aims to solve the problem with an exact algorithm but fails to present an algorithm with practicable running time. The dissertation of \cite{kleff:2019}, which also addresses the TDSP, extends this work and presents an exact algorithm and a heuristic for the TDSRP. They allow parking only at dedicated nodes in the network for which they introduce the term \emph{no-break-en-route policy}. The number of breaks between customers is limited to one break. The master's thesis of \cite{bomsdorf:2020} targets the TDSRP with no-break-en-route-policy in its non-time-dependent version. They restrict the number of types of driving time constraints to two (TDSRP-2B) and provide an exact solution using mixed integer linear programming and, additionally, a heuristic approach. The exact solution fails to provide practical running times and the heuristic approach still suffers from long running times in realistic scenarios. They also briefly present an extension to solve a time-dependent variant of the problem (TD-TDSRP-2B). The TDSRP-2B in contrast to the TDSP is already related to the long-haul truck driver routing problem (LH-TDRP) of our work. The LH-TDRP closely resembles the problem of routing between customers in the TDSRP-2B with regard to parking areas (no-breaks-en-route-policy) and driving time constraints.

The LH-TDRP itself or variations of it were also addressed in previous work, although using different terminology. The work of \cite{kleff:2017} restricts the LH-TDRP by only allowing one break on a route, but includes traffic predictions using time-dependent road networks. It uses a version of core contraction hierarchies where all the parking nodes are part of the core, which we will revisit later in this work. The authors of \cite{tuin:2018} also allow only one break on a route and combine their approach with temporary driving bans and road closures. First, they introduce a multi-label version of Dijkstra's algorithm. Second, they present a heuristic version using time-dependent CH with an improved runtime in comparison.

Other publications address different complex real-world scenarios which present an extension to the SPP. As in this work, most of them strive to integrate the additional requirements into CH or other speed-up techniques which exploit the hierarchy of road networks \cite{bast:2015}. In \cite{kleff:2020}, the authors consider a different scenario in which parking on a route becomes important. The work also addresses temporary driving bans and road closures. In this scenario, it may occur that a driver has to wait for a driving ban or a road closure to be lifted. This waiting time shall be spent in a designated parking area and not en-route. The necessary number of breaks on a route arises from the road network and query, therefore the authors do not restrict the number of possible breaks on a route as done in other publications. The work also includes a rating of parking areas. It therefore searches for pareto-optimal solutions with regard to travel time and ratings of the used parking areas. The approach is too slow to be used in practice, but it uses a combination of the A* algorithm and CH-Potentials which we also will revisit later. Integrating traffic predictions, respectively time-dependent road networks with CH is addressed by \cite{batz:2009,batz:2013}. Turn costs are addressed by \cite{geisberger:2011}. Mitigating the complex task of integrating additional information into CH is addressed by \cite{strasser:2021} by building a tight potential based on CH which then is used in combination with the A* algorithm. They use the same principle to address live traffic and extend the CH variant to a customizable contraction hierarchy \cite{dibbelt:2015} to address temporary driving bans.

TODO: \cite{mayerle:2020} \cite{vital:2019}