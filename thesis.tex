\documentclass{thesisclass}
% Based on thesisclass.cls of Timo Rohrberg, 2009
% ----------------------------------------------------------------
% Thesis - Main document
% ----------------------------------------------------------------


%% -------------------------------
%% |  Information for PDF file   |
%% -------------------------------
\hypersetup{
 pdfauthor={Max Oesterle},
 pdftitle={Efficient Long-Haul Truck Driver Routing},
 pdfsubject={?},
 pdfkeywords={?}
}


%% ---------------------------------
%% | Information about the thesis  |
%% ---------------------------------

\newcommand{\myname}{Max Oesterle}
\newcommand{\mytitle}{Efficient Long-Haul Truck Driver Routing}
\newcommand{\myinstitute}{Institute of Theoretical Informatics}

\newcommand{\reviewerone}{Dr. rer. nat. Torsten Ueckerdt}
\newcommand{\reviewertwo}{?}
\newcommand{\advisor}{Tim Zeitz}
\newcommand{\advisortwo}{Alexander Kleff}
\newcommand{\advisorthree}{Frank Schulz}

\newcommand{\timestart}{15th January 2022}
\newcommand{\timeend}{15th July 2022}


%% ---------------------------------
%% | Commands                      |
%% ---------------------------------

\newtheorem{definition}{Definition} \numberwithin{definition}{chapter}
\newtheorem{theorem}[definition]{Theorem}
\newtheorem{lemma}[definition]{Lemma}
\newtheorem{corollary}[definition]{Corollary}
\newtheorem{conjecture}[definition]{Conjecture}

%% --------------------------------
%% | Settings for word separation |
%% --------------------------------
% Help for separation:
% In german package the following hints are additionally available:
% "- = Additional separation
% "| = Suppress ligation and possible separation (e.g. Schaf"|fell)
% "~ = Hyphenation without separation (e.g. bergauf und "~ab)
% "= = Hyphenation with separation before and after
% "" = Separation without a hyphenation (e.g. und/""oder)

% Describe separation hints here:
\hyphenation{
% Pro-to-koll-in-stan-zen
% Ma-na-ge-ment  Netz-werk-ele-men-ten
% Netz-werk Netz-werk-re-ser-vie-rung
% Netz-werk-adap-ter Fein-ju-stier-ung
% Da-ten-strom-spe-zi-fi-ka-tion Pa-ket-rumpf
% Kon-troll-in-stanz
}


%% ------------------------
%% |    Including files   |
%% ------------------------
% Only files listed here will be included!
% Userful command for partially translating the document (for bug-fixing e.g.)
\includeonly{
titlepage,
introduction,
preliminaries,
problem,
algorithm,
evaluation,
conclusion,
appendix
}

\usepackage{mathtools}
\DeclarePairedDelimiter{\floor}{\lfloor}{\rfloor}
\DeclarePairedDelimiter{\ceil}{\lceil}{\rceil}


%%%%%%%%%%%%%%%%%%%%%%%%%%%%%%%%%
%% Here, main documents begins %%
%%%%%%%%%%%%%%%%%%%%%%%%%%%%%%%%%
\begin{document}

% Remove the following line for German text
\selectlanguage{english}

\frontmatter
\pagenumbering{roman}
%% titlepage.tex
%%

% coordinates for the bg shape on the titlepage
\newcommand{\diameter}{7}
\newcommand{\xone}{-15}
\newcommand{\xtwo}{160}
\newcommand{\yone}{15}
\newcommand{\ytwo}{-253}

\begin{titlepage}
% bg shape
\begin{tikzpicture}[overlay]
\draw[color=gray]  
 		 (\xone mm, \yone mm)
  -- (\xtwo mm, \yone mm)
 arc (90:0:\diameter mm) 
  -- (\xtwo mm + \diameter mm , \ytwo mm) 
	-- (\xone mm + \diameter mm , \ytwo mm)
 arc (270:180:\diameter mm)
	-- (\xone mm, \yone mm);
\end{tikzpicture}
	\begin{textblock}{10}[0,0](4,2.5)
		\includegraphics[width=.3\textwidth]{logos/KITLogo.pdf}
	\end{textblock}
        \begin{textblock}{10}[0,0](14.5,2.45)
		\includegraphics[width=.15\textwidth]{logos/algoLogo.pdf}
	\end{textblock}
	\changefont{phv}{m}{n}	% helvetica	
	\vspace*{3.75cm}
	\begin{center}
		\Huge{\mytitle}
		\vspace*{2.25cm}\\
		\Large{
			\iflanguage{english}{Master Thesis of}			
												  {Diplomarbeit\\von}
		}\\
		\vspace*{1cm}
		\huge{\myname}\\
		\vspace*{1cm}
		\Large{
			\iflanguage{english}{At the Department of Informatics}			
													{An der Fakult\"at f\"ur Informatik}
			\\
			\myinstitute
		}
	\end{center}
	\vspace*{1cm}
\Large{
\begin{center}
\begin{tabular}[ht]{l c l}
  % Gutachter sind die Professoren, die die Arbeit bewerten. 
  \iflanguage{english}{Reviewers}{Erstgutachter}: & \hfill & \reviewerone\\
  \iflanguage{english}{}{Zweitgutachter:} & \hfill & \reviewertwo\\
  \iflanguage{english}{Advisors}{Betreuende Mitarbeiter}: & \hfill & \advisor\\
  \iflanguage{english}{}{} & \hfill & \advisortwo\\
  % Der zweite betreuende Mitarbeiter kann weggelassen werden. 
\end{tabular}
\end{center}
}


\vspace{2cm}
\begin{center}
\large{\iflanguage{english}{Time Period}{Bearbeitungszeit}: \ \timestart{} \ -- \ \timeend}
\end{center}


\begin{textblock}{10}[0,0](4,16.8)
\tiny{ 
	\iflanguage{english}
		{KIT -- The Research University in the Helmholtz Association}
		{KIT -- Die Forschungsuniversit\"at in der Helmholtz-Gemeinschaft}
}
\end{textblock}

\begin{textblock}{10}[0,0](14,16.75)
\large{
	\textbf{www.kit.edu} 
}
\end{textblock}

\end{titlepage}

\blankpage

%% -------------------------------
%% |   Statement of Authorship   |
%% -------------------------------

\thispagestyle{plain}

\vspace*{\fill}

\centerline{\textbf{Statement of Authorship}}

\vspace{0.25cm}

Ich versichere wahrheitsgemäß, die Arbeit selbstständig verfasst, alle benutzten Hilfsmittel vollständig und genau angegeben und alles kenntlich gemacht zu haben, was aus Arbeiten anderer unverändert oder mit Abänderungen entnommen wurde sowie die Satzung des KIT zur Sicherung guter wissenschaftlicher Praxis in der jeweils gültigen Fassung beachtet zu haben.

\vspace{2.5cm}

\hspace{0.25cm} Karlsruhe, \today

\vspace{2cm}

\blankpage

%% -------------------
%% |   Abstract      |
%% -------------------

\thispagestyle{plain}

\begin{addmargin}{0.5cm}

	\centerline{\textbf{Abstract}}

	A short summary of what is going on here.

	\vskip 2cm

	\centerline{\textbf{Deutsche Zusammenfassung}}

	Kurze Inhaltsangabe auf deutsch.

\end{addmargin}

\blankpage

%% -------------------
%% |   Directories   |
%% -------------------

\tableofcontents
\blankpage

%% -----------------
%% |   Main part   |
%% -----------------

\mainmatter
\pagenumbering{arabic}
%% introduction.tex
%%

%% ==============================
\chapter{Introduction}
\label{ch:introduction}
%% ==============================

This chapter should contain
\begin{enumerate}
  \item A short description of the thesis topic and its background.
  \item An overview of related work in this field.
  \item Contributions of the thesis.
  \item Outline of the thesis.
\end{enumerate}
% !TeX root = thesis.tex
%% preliminaries.tex
%%

%% ==============
\chapter{Preliminaries\label{ch:preliminaries}}
%% ==============
In this chapter, we introduce our basic notations and discuss important algorithmic concepts on which the work of this thesis is based.

We define a weighted, directed graph $G$ as a tuple $G=(V,E,\mathfunction{len})$. $V$ is the set of nodes and $E$ the set of edges $(u,v) \subseteq V \times V$ between those nodes. The function $\mathfunction{len}$ is the weight function $\mathfunction{len}\colon E \rightarrow \mathbb{R}_{\ge 0}$ which assigns each edge a non-negative weight which we often also call length of an edge. A path $p$ in $G$ is defined as a sequence of nodes $p = \langle v_0,v_1,...,v_k \rangle$ with $(v_i,v_{i+1}) \in E$. For simplicity, we will reuse the same function $\mathfunction{len}$ which we use to denote the length of an edge, to denote the length of a path $p$ in $G$. The length of a path $\mathfunction{len}(p)$ is defined by the sum of the weights of the edges on the path $\mathfunction{len}(p) = \sum_{i=0}^{k-1}{\mathfunction{len}((v_i,v_{i+1}))}$. A path must not necessarily be simple, i.e. nodes can appear multiple times in the same path.

Given two nodes $s$ and $t$ in a graph, we denote the shortest distance between them as $\distance(s,t)$. The shortest distance between two nodes is the minimum length of a path between them. The problem of finding the shortest distance and an associated path between two nodes in a graph is called the shortest path problem which we often abbreviate as SPP. The problem of finding the shortest path between nodes in a road network can be formalized as solving the SPP on a weighted, directed graph. Each edge of the graph represents a road and each node represents an intersection. Unless stated otherwise, the length $\len$ of an edge $(u,v)$ will always correspond to the time it takes to travel from $u$ to $v$ on the road which the edge represents. A solution of the SPP then yields the shortest time between two intersections in the road network and the associated path between them.

Dijkstra's algorithm, published in 1959, solves the SPP \cite{dijkstra:1959}. It operates on the graph $G$ without any additionally information or precomputed data structures. It maintains a queue $Q$ of nodes with ascending tentative distance from the starting node $s$ and two arrays, a distance value $d[v]$ and a predecessor node $pred[v]$ for each node. At the beginning, $Q$ only contains the start node $s$ with the distance zero. The two arrays are initialized with $d[v]=\infty$ and $pred[v]=\bot$ except for $d[s]=0$ and $pred[s]=s$. Iteratively, the node $u$ with the minimum distance is removed from $Q$ and each outgoing edge $(u,v) \in E$ of $u$ is \emph{relaxed}. We call this process \emph{settling} a node $u$. Relaxing an edge $(u,v)$ consists of three steps: First, the sum $d[u] + \len((u,v))$ is calculated. Second, it is tested if the distance $d[v]$ can be improved by choosing $u$ as a predecessor. Finally, if that is the case, the queue key of the node $v$ is decreased. If $v$ is not contained in $Q$ yet, the node is inserted into $Q$. The search can be stopped if the target node $t$ was removed from the queue \cite{dijkstra:1959}.

For many practical applications and for large graphs, Dijkstra's algorithm is too slow. A common extension is the A* algorithm \cite{hart:1968}. A* uses a \emph{heuristic} which yields a lower bound for the distance from each node to the target node to direct the search towards the goal. With a tight heuristic, A* can significantly reduce the search space, i.e., the amount of nodes it touches during the search in comparison to Dijkstra's algorithm. In the route planning context, the term \emph{potential} is often used for the heuristic. We will denote the potential of a node $v$ to a node $t$ with $\potential_t(v)$.

To further reduce the search space, it is possible to run a \emph{bidirectional} search. A bidirectional search to solve the SPP from $s$ to $t$ on a graph $G$ consists of a forward search and a backward search. The forward search operates on $G$ with start node $s$ and target node $t$ and maintains a forward queue $\overrightarrow{Q}$, a forward distance array $\overrightarrow{d}[v]$, and a forward predecessor array $\overrightarrow{pred}[v]$. The backward search operates on a backward graph $\overleftarrow{G}$ with start node $t$ and target node $s$ and maintains a backward queue $\overleftarrow{Q}$, a backward distance array $\overleftarrow{d}[v]$, and a backward predecessor array $\overleftarrow{pred}[v]$. The backward graph is defined as the graph $G$ with inverted edges, i.e, $\overleftarrow{G} = (V,\overleftarrow{E},\overleftarrow{\mathfunction{len}})$ with $\overleftarrow{E} = \{(v,u) \in V \times V \mid (u,v) \in E\}$ and $\overleftarrow{\mathfunction{len}}(u,v) = \mathfunction{len}(v,u)$. Additionally, an array $d[v]$ is maintained for combined tentative distances of forward and backward search and is initialized with $\infty$ for all nodes. A value $d[v]$ constitutes the shortest known distance for an $s$-$t$ path using $v$.

Forward and backward search now alternately settle a node $v$. If $v$ was settled by both searches, forward and backward search met at $v$. The value $d[v]$ is updated to the sum of $\overrightarrow{d}[v] + \overleftarrow{d}[v]$ if it is an improvement over the old value, i.e., it yields a shorter distance for an $s$-$t$ path via $v$.

The bidirectional search only yields an advantage over a unidirectional search if the two searches are stopped earlier than in a unidirectional search. If not, the bidirectional search would simply execute the work of an $s$-$t$ search twice. Therefore, stronger stopping criteria are introduced. When introducing a strong stopping criterion for a bidirectional A* search, the stopping criterion also depends on the potential being used. The work of \cite{goldberg:2005a} introduces a potential and stopping criterion which leads to an improvement over a unidirectional A* search. If a forward potential $\overrightarrow{\potential}_t$ and a backward potential $\overleftarrow{\potential}_s$ is used for the forward and the backward search, the search can be stopped when the minimum key of the forward queue $\minKey(\overrightarrow{Q})$ or the minimum key of the backward queue $\minKey(\overrightarrow{Q})$ is greater than the currently known shortest distance between $s$ and $t$. If $\overrightarrow{\potential}_t + \overleftarrow{\potential}_s \equiv const.$ holds for the potentials, then the search can be stopped when $\minKey(\overrightarrow{Q}) + \minKey(\overleftarrow{Q})$ exceeds the shortest know distance between $s$ and $t$.

\section{Contraction Hierarchies\label{sec:ch}}
Road networks have strong hierarchies since some roads are more important than others. For example, a highway allows high average speeds and therefore is a preferred connection between points in a road network in comparison to smaller roads. Contraction Hierarchies \cite{geisberger:2012} are a speed-up technique which exploits these hierarchies.

Contraction Hierarchies (CH) use a two phase approach. In the preprocessing phase, the CH is constructed given a graph $G=(V,E,\len)$ and a node order which sorts the nodes by importance. For example, an important node of a road network might be a node in a highway interchange, an unimportant node might be the head of a dead end. We denote the CH as $\ch$. The node order can be computed by searching for unimportant nodes \cite{geisberger:2012} or for important nodes first \cite{abraham:2012,delling:2014}. The nodes then are sorted into multiple levels of increasing importance. Two nodes of the same level must not have an edge between them. We obtain the CH $\ch$ by iteratively contracting the least important node according to the node order by adding shortcut edges between its neighbors.

An outgoing edge of a node to another node of a higher level is called an \emph{upward} edge, the opposite is called a \emph{downward} edge. A path which consists of only upward edges is called an \emph{up-path}, a path which consists of only downward edges is called a \emph{down-path}. Finally, a path which consists of an up-path, followed by a down-path, is called an \emph{up-down-path}. The node on an up-down-path with the highest level, i.e. the node which separates up-path and down-path, is called the \emph{middle node} $m$ of the path.

For every shortest path between node $s,t \in V$, there exists an up-down $s$-$m$-$t$ path in $\ch$ of the exact same length \cite{geisberger:2012}. Therefore, we can restrict the search to finding this exact path. We run a bidirectional search from $s$ and $t$. The forward search from $s$ may only use upward edges and finds the subpath $s$-$m$, the backward search from $t$ may only use the inverted downward edges and finds the subpath $m$-$t$. The search is stopped if the minimum queue key of both queues of forward and backward search is greater or equal to the tentative minimum distance $\distance(s,t)$ \cite{geisberger:2012}.


\section{Core Contraction Hierarchies\label{sec:core_ch}}
The core contraction hierarchy presents a compromise between constructing a full CH $\ch$ and running a bidirectional search on $G$. The core CH $\corech$ is obtained by stopping the iterative contraction of nodes of increasing importance during the construction of the CH early. This leads to a set $C \subseteq V$ of so-called core nodes which are uncontracted. The set of nodes $V$ therefore is separated into a set of core nodes $C$ and a set of contracted nodes $\chv = V \setminus C$. The graph $\ch = (\chv,\che,\mathfunction{len})$ therefore is a valid contraction hierarchy. It contains only contracted nodes and (shortcut) edges between those nodes. The graph $G_{C} = (V \setminus \chv,E \setminus \che,\mathfunction{len})$ is called the core graph.

The core CH query again is a bidirectional search from $s$ and $t$. The query represents a normal CH query while settling nodes $v \in \chv$, i.e., it consists of a forward search from $s$ using only upward edges and a backward search from $t$ using only inverted downward edges. If a search reaches an uncontracted core node, it considers all outgoing edges. Thus, the core CH query can be characterized as a CH query in $\ch$ which transitions into a full bidirectional search if it reaches the core $G_C$. We can use the same stopping criterion as for the CH query since the stopping criterion for a CH is more conservative than the stopping criterion for a pure bidirectional search.


\section{CH-Potentials\label{sec:ch_pot}}
CH-Potentials \cite{strasser:2021b} are an extension of CH to efficiently calculate distances $\distance(v,t)$ for many nodes $v \in V$ to a fixed node $t$. CH-Potentials are based on PHAST \cite{delling:2011} which itself is an extension of CH to efficiently run all-to-one queries.

PHAST runs in two steps. The first step is a backward one-to-all search from $t$ using only inverted downward edges. This is the same as running the backward search of a CH query from $t$ without any stopping criterion. We obtain an array $B$ where $B[v]$ is the length of the shortest down path from $v$ to $t$ or $B[v]=\infty$ if there is no such path. We then iteratively calculate the distance $\distance(v,t)$ of nodes of the same level, starting at the highest level. This is possible since nodes of the same level must not have edges between them. To obtain the distances of nodes $u$ of the next lower level, we find the minimum $\min(B[u], \min_v(\len(u,v) + \distance(v,t)))$ for all upward edges $(u,v)$ of the node $v$. The distances $\distance(v,t)$ were already computed in the previous iteration.

We can use PHAST as to compute potentials if we run its two steps beforehand as a preprocessing step and then look up the respective distances. This preprocessing would be slow since it computes the distances to all nodes. CH-Potentials mitigate this problem by computing the results of the second PHAST step lazily while the first step remains the same. We do not calculate the distances of all nodes beforehand. To compute the potential of a node $u$, we recursively compute the potential for all nodes $v$ which are connected to $u$ by upward edges $(u,v) \in \che$. We then can compute the distance to $t$ as in the PHAST algorithm by finding $\min(B[u], \min_v(\len(u,v) + \distance(v,t)))$. We save all the calculated distances $\distance(v,t)$ in order to not compute any distance value twice.

CH-Potentials can be optimized further as \cite{strasser:2021b} describes in detail.
\chapter{Problem and Definitions}\label{chapter:problem_definitions}
Truck drivers have to follow regulations regarding the maximum time they are allowed to drive without taking a break. Therefore, on longer routes planning becomes necessary. We propose an extension of the shortest path problem which accounts for driving time limits and mandatory breaks, denoted as the long-haul truck driver routing problem. It can be formalized as follows:

Let $G=(V,E,\mathfunction{len})$ be a graph and $s$ and $t$ nodes with $s,t \in V$. We extend the graph with a set $P \subseteq V$ of parking nodes. Additionally, we introduce a set $\restrset$ of driving time constraints $\restr_i$. Each driving time constraint is defined by a maximum permitted driving time $\restr_i^d$ and a break time $\restr_i^b$. We assume an order among the constraints and that both the driving time and the break time correspond to this order, i.e. $i<j \implies c_i^d < c_j^d \wedge c_i^b < c_j^b$. Before exceeding a driving time of $\restr_i^d$, the driver must stop and break for a time of at least $\restr_i^b$. Afterwards, the driver is allowed to drive for a maximum time of $\restr_i^d$ again without stopping. Breaks can only take place at nodes $v \in P$.

A route $\route$ from $s$ to $t$ includes the path of visited nodes $p = \langle s=v_0,v_1,\dots,t=v_k \rangle$ and a break time function $\breakTime\colon p \rightarrow \{0,\restr_1^d,\dots,\restr_{|\restrset|}^d\}$ at each node $i$. For non-parking nodes $v_i \notin P$, the break time must be zero. We also define the \breakTime\ of an entire route $\route$ on $p$ as $\breakTime(\route) = \sum_{i=0}^{k-1}{\breakTime((v_i,v_{i+1}))}$.

\begin{definition}[Valid Route]
	A valid route $p = \langle s=v_0,v_1,...,t=v_k \rangle$ must comply with all driving time constraints in $\restrset$.A path complies with a specific driving time constraint $\restr \in \restrset$ if there is no subpath $p'$ between two nodes $u,w \in P' = \{s,t\} \cup \{v_i \in p \mid \breakTime(v_i) \ge \restr^b\}$ on the path which exceeds the driving time limit $\len(p') > \restr^d$ and has no third node $v_i \in P'$ in between $u$ and $w$.
\end{definition}

We differentiate between the driving time and the travel time of a route.

\begin{definition}[Driving Time of a Route]
	The driving time $\concretedt(\route)$ of a route $\route$ is the length $\len(p)$ of the path $p = \langle s=v_0,v_1,...,t=v_k \rangle$ of the route.
\end{definition}

\begin{definition}[Travel Time of a Route]
	The travel time $\concretett(\route)$ of a route $\route$ is the sum of driving time $\concretedt(\route)$ and break time  $\breakTime(p)$ on its path.
\end{definition}

\begin{definition}[Shortest Route]
	A route between two nodes $s$ and $t$ is called a shortest route if it is valid and there exists no different valid route between $s$ and $t$ with a smaller travel time.
\end{definition}

The shortest travel time between two nodes $s$ and $t$, i.e., the travel time of the shortest route between them is denoted as $\traveltime(s,t)$. Accordingly, $\drivingtime(s,t)$ denotes the driving time of the shortest route between $s$ and $t$. In general, the driving time $\drivingtime(s,t)$ and the distance $\distance(s,t)$ between two nodes are not equal.

The long-haul truck driver routing problem now can be defined as follows.

\begin{namedproblem}
	\problemtitle{\textsc{Long-Haul Truck Driver Routing}}
	\probleminput{A graph $G=(V,E,\mathfunction{len})$, a set of parking nodes $P \subseteq V$, a set of driving time constraints $\restrset$, and start and target nodes $s,t \in V$}
	\problemquestion{Find the shortest valid route $r$ from $s$ to $t$ in $G$.}
\end{namedproblem}

In many practical applications, the number of different driving time constraints is limited to only one or two constraints, i.e., $|\restrset| = 1$ or  $|\restrset| = 2$. Therefore, we will often only consider two special cases.
%% content.tex
%%

%% ==============
\chapter{Algorithm}
\label{ch:Algorithm}
%% ==============
In this chapter, we introduce a labeling algorithm which solves the long-haul truck driver routing problem. We then describe extensions of the base algorithm to achieve better runtimes on road networks in practice. At first, we will restrict the problem to one driving time constraint for simplicity and drop that constraint later.

\section{Dijkstra's Algorithm with One Driving Time Constraint}
Dijkstra's algorithm solves the shortest path problem by maintaining a queue $Q$ of nodes with ascending tentative distance from the starting node $s$, and iteratively settling the node with the smallest distance. When Dijkstra settles a node $u$, it tests if the distance to all neighbor nodes $v$ with $(u,v) \in E$ can be improved by choosing the current node as a predecessor. We say it \emph{relaxes} all edges $(u,v) \in E$. The search can be stopped if the target node $t$ was removed from the queue. TODO REFERENCE PROOF

We will adapt Dijkstra's algorithm for solving the long-haul truck driver routing problem with one driving time constraint $r$. While Dijkstra's algorithm manages a queue of nodes and assigns each node one tentative distance, our algorithm manages a queue $Q$ of labels and a set $L(v)$ of labels for each node $v \in V$.

Labels represent a possible path, respectively a possible solution for a query from $s$ to the node they belong to. A label $l$ contains

\begin{itemize}
	\item $d_0(l)$, the total travel time from the starting node $s$
	\item $d_1(l)$, the driving time since the last pause
	\item $pred(l)$, its preceding label
\end{itemize}

\subsection{Settling a Label}
In contrast to Dijkstra, the search \emph{settles} a label $l \in L(u)$ in each iteration instead of a node $u$. When settling a label, the search first removes $l$ from the queue. Similar to Dijkstra, it then relaxes all edges $(u,v) \in E$ with $l \in L(u)$ as shown in fig. \ref{alg:settle_next_label}.

\begin{figure}[hbtp]
	\setlength{\interspacetitleruled}{0pt}%
	\setlength{\algotitleheightrule}{0pt}%
	\begin{algorithm*}[H]
		\SetFuncSty{textsc}
		\DontPrintSemicolon
		\SetKwData{Q}{Q}
		\SetKwData{L}{L}

		\SetKwFunction{settleNextLabel}{settleNextLabel}
		\SetKwFunction{relaxEdge}{relaxEdge}
		\SetKwFunction{queueDeleteMin}{deleteMin}

		\SetKwProg{Pn}{Procedure}{:}{\KwRet}
		\Pn{\settleNextLabel{}}{
			$l \leftarrow$ \Q.\queueDeleteMin{} \;
			\BlankLine

			\ForAll{ $(u,v) \in E$ }
			{
				\relaxEdge{$(u,v),l$}
			}
		}
	\end{algorithm*}
	\setlength{\interspacetitleruled}{2pt}%
	\setlength{\algotitleheightrule}{\algotitleheightruledefault}%

	\caption{\label{alg:settle_next_label}Settling a label $l \in L(u)$ removes the label from the queue and relaxes all the outgoing edges of $u$.}
\end{figure}

Relaxing an edge consists of the three steps label \emph{propagation}, \emph{pruning} and \emph{dominance} checks.

\paragraph{Label Propagation.}
Labels can be propagated along edges. Let $l \in L(u)$ be a label at $u$ and $(u,v) = e \in E$, then $l$ can be propagated to $v$ resulting in a label $l^{'}$ with $d_0(l^{'}) = d_0(l) + \omega(e)$, $d_1(l^{'}) = d_1(l) + \omega(e)$, and $pred(l^{'}) = l$.

\paragraph{Label Pruning.}
After propagating a label, we discard the label if it violates the driving time constraint $r$. That is, if $d_1(l) \ge r_d$.


\paragraph{Label Dominance}
In general, it is no longer clear when a label presents a better solution than another label since it now contains two distance values. A label $l$ at a node $v$ might represent a path with a shorter travel time from $s$ to $v$ than another label $m$, but a shorter remaining driving time budget $r_d - d_1(l)$. The label $l$ yields a better solution for a query $s$-$v$, but this does not imply that it is part of a better solution for a query from $s$ to $t$. It might not even yield a path to $t$ at all while $m$ reaches the target due to the greater remaining driving time budget. In one case we can proof that a label $l \in L(v)$ cannot yield a better solution than a label $m \in L(v)$. We say $m$ \emph{dominates} $l$.

\begin{definition}[Label Dominance]
	A label $l \in L(v)$ dominates another label $m \in L(v)$ if $d_0(m) \ge d_0(l)$ and $d_1(m) \ge d_1(l)$.
\end{definition}

If a label $l \in L(v)$ is dominated by another label $m \in L(v)$, then $m$ represents a path from $s$ to $t$ with a shorter or equal total travel time and longer remaining driving time budget until the next pause. Therefore, in each solution which uses the label $l$, $l$ can trivially be replaced by the label $m$. The solution will still comply with the driving time constraint $r$ and yield a shorter or equal total travel time, so we are allowed to simply discard dominated labels in our search.

\begin{definition}[Pareto-Optimal Label]
	A label $l \in L(v)$ is pareto-optimal if it is not dominated by any other label $m \in L(v)$.
\end{definition}

We will only add a label $l$ to a label set $L(v)$ if it is pareto-optimal. Labels $m \in L(v)$ are then removed from $L(v)$ if $l$ dominates them. $L(v)$ therefore is the set of pareto-optimal solutions at $v$. In fig. \ref{alg:remove_dominated} we define the procedure \textsc{removeDominated($l$)} as a label set operation.

\begin{figure}[hbtp]
	\setlength{\interspacetitleruled}{0pt}%
	\setlength{\algotitleheightrule}{0pt}%
	\begin{algorithm*}[H]
		\SetFuncSty{textsc}
		\SetKwFor{ForAll}{forall}{do}
		\DontPrintSemicolon
		\SetKwData{L}{L}

		\SetKwFunction{removeDominated}{removeDominated}
		\SetKwFunction{queueRemove}{remove}

		\SetKwProg{Pn}{Procedure}{:}{\KwRet}
		\Pn{\removeDominated{$l$}}{
			\ForAll{$m \in L$}
			{
				\If{\text{$l$ dominates $m$}}{
					\L.\queueRemove($m$)\;
				}
			}
		}
	\end{algorithm*}
	\setlength{\interspacetitleruled}{2pt}%
	\setlength{\algotitleheightrule}{\algotitleheightruledefault}%

	\caption{\label{alg:remove_dominated}The operation which is defined on a label set $L$, removes all labels from the set which are dominated by the label $l$.}
\end{figure}

We now use \textsc{removeDominated} to define the procedure \textsc{relaxEdge$'$} as described in fig. \ref{alg:relax_edge_no_p} which propagates a label along an edge and updates the neighbor node's label set if necessary.

\begin{figure}[hbtp]
	\setlength{\interspacetitleruled}{0pt}%
	\setlength{\algotitleheightrule}{0pt}%
	\begin{algorithm*}[H]
		\SetFuncSty{textsc}
		\DontPrintSemicolon
		\SetKwData{Q}{Q}
		\SetKwData{L}{L}

		\SetKwFunction{relaxEdgeNoP}{relaxEdge$'$}
		\SetKwFunction{queueDeleteMin}{deleteMin}

		\SetKwData{Q}{Q}
		\SetKwData{dist}{d}
		\SetKwData{L}{L}
		\SetKwData{pred}{pred}
		\SetKwArray{ds}{ds}
		\SetKwFunction{queueDeleteMin}{deleteMin}
		\SetKwFunction{queueInsert}{insert}
		\SetKwFunction{queueMin}{min}
		\SetKwFunction{queueMinKey}{minKey}
		\SetKwFunction{queueDecreaseKey}{decreaseKey}
		\SetKwFunction{queueContains}{contains}
		\SetKwFunction{listInsert}{insert}
		\SetKwFunction{removeDom}{removeDominated}

		\SetKwProg{Pn}{Procedure}{:}{\KwRet}
		\Pn{\relaxEdgeNoP{(u,v), l}}{
			\If{$d_1(l) + \omega(u,v) < r_d$}
			{
				$l^{'} \leftarrow \{(d_0(l) + \omega(u,v), d_1(l) + \omega(u,v)), l\}$\;
				\If{$l^{'}$ is not dominated by any label in \L{$v$}}
				{
					\L{$v$}.\removeDom{$l^{'}$} \;
					\L{$v$}.\queueInsert{$l^{'}$} \;
					\Q.\queueInsert{$l'$}
				}
			}
		}
	\end{algorithm*}
	\setlength{\interspacetitleruled}{2pt}%
	\setlength{\algotitleheightrule}{\algotitleheightruledefault}%

	\caption{\label{alg:relax_edge_no_p}Relaxing an edge $(u,v) \in E$ when settling a label $l \in L(u)$.}
\end{figure}

\subsection{Parking at a Node}
The procedure \textsc{relaxEdge$'$} does not account for parking nodes. When propagating a label $l \in L(u)$ along an edge $(u,v) \in E$ and $v \in P$ then we have to consider pausing at $v$. Since we do not know if pausing at $v$ or continuing without a pause is the better solution, we generate both labels and them to label set $L(v)$ and the queue $Q$ as defined in fig. \ref{alg:relax_edge}.

\begin{figure}[hbtp]
	\setlength{\interspacetitleruled}{0pt}%
	\setlength{\algotitleheightrule}{0pt}%
	\begin{algorithm*}[H]
		\SetFuncSty{textsc}
		\DontPrintSemicolon
		\SetKwData{Q}{Q}
		\SetKwData{L}{L}

		\SetKwFunction{relaxEdge}{relaxEdge}
		\SetKwFunction{queueDeleteMin}{deleteMin}

		\SetKwData{Q}{Q}
		\SetKwData{L}{L}
		\SetKwData{D}{D}
		\SetKwData{pred}{pred}
		\SetKwArray{ds}{ds}
		\SetKwFunction{queueDeleteMin}{deleteMin}
		\SetKwFunction{queueInsert}{insert}
		\SetKwFunction{queueMin}{min}
		\SetKwFunction{queueMinKey}{minKey}
		\SetKwFunction{queueDecreaseKey}{decreaseKey}
		\SetKwFunction{queueContains}{contains}
		\SetKwFunction{listInsert}{insert}
		\SetKwFunction{removeDom}{removeDominated}

		\SetKwProg{Pn}{Procedure}{:}{\KwRet}
		\Pn{\relaxEdge{(u,v), l}}{
			\If{$d_1(l) + \omega(u,v) < r_d$}
			{
				\D.\queueInsert{$(d_0(l) + \omega(u,v), d_1(l) + \omega(u,v), l)$}\;

				\If{$v \in P$}
				{
					\D.\queueInsert{$(d_0(l) + \omega(u,v) + d_p, 0, l)$}\;
				}

				\ForAll{ $l' \in D$ }
				{
					\If{$l'$ is not dominated by any label in \L{$v$}}
					{
						\L{$v$}.\removeDom{$l^{'}$} \;
						\L{$v$}.\queueInsert{$l^{'}$} \;
						\Q.\queueInsert{$l'$}
					}
				}
			}
		}
	\end{algorithm*}
	\setlength{\interspacetitleruled}{2pt}%
	\setlength{\algotitleheightrule}{\algotitleheightruledefault}%

	\caption{\label{alg:relax_edge}Relaxing an edge $(u,v) \in E$ when settling a label $l \in L(u)$ with regard to parking nodes.}
\end{figure}

\subsection{Initialization and Stopping Criterion}
We initialize the label set $L(s)$ of $s$ and the queue $Q$ with a label which only contains distances of zero and a dummy element as a predecessor. We stop the search when $t$ was removed from $Q$. The definition of the final algorithm \ref{alg:CSP} \textsc{Dijkstra+1DTC} is now trivial.

\begin{algorithm}[bt]
	\caption{\textsc{Dijkstra+1DTC}}\label{alg:CSP}

	% Some settings
	\DontPrintSemicolon %dontprintsemicolon
	\SetFuncSty{textsc}
	\SetKwFor{ForAll}{forall}{do}
	\SetKw{Return}{return}

	% Declaration of data containers and functions
	\SetKwData{Q}{Q}
	\SetKwData{dist}{d}
	\SetKwData{L}{L}
	\SetKwData{pred}{pred}
	\SetKwArray{ds}{ds}
	\SetKwFunction{queueDeleteMin}{deleteMin}
	\SetKwFunction{queueInsert}{insert}
	\SetKwFunction{queueMin}{min}
	\SetKwFunction{queueMinKey}{minKey}
	\SetKwFunction{queueDecreaseKey}{decreaseKey}
	\SetKwFunction{queueContains}{contains}
	\SetKwFunction{listInsert}{insert}
	\SetKwFunction{removeDom}{removeDominated}
	\SetKwFunction{settleNextNode}{settleNextNode}

	% Algorithm interface
	\KwIn{Graph $G=(V,E,\omega)$, set of parking nodes $P \subseteq V$, set of driving time constraints $R$, start and target nodes $s,t \in V$}
	\KwData{Priority queue \Q, per node set \L{$v$} of labels for all $v \in V$}
	\KwOut{Shortest travel time from $s$ to $t$ and corresponding $s$-$t$-path given by the predecessors of the label $l_t \in L(t)$}

	% The algorithm
	\BlankLine
	\tcp{Initialization}
	\Q.\queueInsert{$(0,0,\bot)$}\;
	\L{$s$}.\queueInsert{$(0,0,\bot)$}\;
	\BlankLine
	\tcp{Main loop}
	\While{\Q is not empty}
	{
		\settleNextNode{}\;

		\If{\text{minimum of $Q$ is label at $t$}}
		{
			\Return\;
		}
	}
\end{algorithm}

\subsection{Correctness}
TODO correctness stopping criterion

\section{A* with One Driving Time Constraint}
TODO dijkstra slow blabla therefore with ch potential just add potential everywhere

\begin{algorithm}[bt]
	\caption{\textsc{A*+1DTC}}\label{alg:CSPPot}

	% Some settings
	\DontPrintSemicolon %dontprintsemicolon
	\SetFuncSty{textsc}
	\SetKwFor{ForAll}{forall}{do}
	\SetKw{Return}{return}

	% Declaration of data containers and functions
	\SetKwData{Q}{Q}
	\SetKwData{dist}{d}
	\SetKwData{L}{L}
	\SetKwData{pred}{pred}
	\SetKwArray{ds}{ds}
	\SetKwFunction{queueDeleteMin}{deleteMin}
	\SetKwFunction{queueInsert}{insert}
	\SetKwFunction{queueMin}{min}
	\SetKwFunction{queueMinKey}{minKey}
	\SetKwFunction{queueDecreaseKey}{decreaseKey}
	\SetKwFunction{queueContains}{contains}
	\SetKwFunction{listInsert}{insert}
	\SetKwFunction{removeDom}{removeDominated}
	\SetKwFunction{settleNextNode}{settleNextNode}

	% Algorithm interface
	\KwIn{Graph $G=(V,E,\omega)$, set of parking nodes $P \subseteq V$, set of driving time constraints $R$, start and target nodes $s,t \in V$, potential $pot()$}
	\KwData{Priority queue \Q, per node set \L{$v$} of labels for all $v \in V$}
	\KwOut{Shortest travel time from $s$ to $t$ and corresponding $s$-$t$-path given by the predecessors of the label $l_t \in L(t)$}

	% The algorithm
	\BlankLine
	\tcp{Initialization}
	\Q.\queueInsert{$pot((0,0,\bot))$}\;
	\L{$s$}.\queueInsert{$(0,0,\bot)$}\;
	\BlankLine
	\tcp{Main loop}
	\While{\Q is not empty}
	{
		\settleNextNode{}\;

		\If{\text{minimum of $Q$ is label at $t$}}
		{
			\Return\;
		}
	}
\end{algorithm}

TODO settlenextlabel is like in fig. \ref{alg:settle_next_label} but relax ege is now as in fig. \ref{alg:relax_edge_a_star}

\begin{figure}[hbtp]
	\setlength{\interspacetitleruled}{0pt}%
	\setlength{\algotitleheightrule}{0pt}%
	\begin{algorithm*}[H]
		\SetFuncSty{textsc}
		\DontPrintSemicolon
		\SetKwData{Q}{Q}
		\SetKwData{L}{L}

		\SetKwFunction{relaxEdge}{relaxEdge}
		\SetKwFunction{queueDeleteMin}{deleteMin}

		\SetKwData{Q}{Q}
		\SetKwData{L}{L}
		\SetKwData{D}{D}
		\SetKwData{pred}{pred}
		\SetKwArray{ds}{ds}
		\SetKwFunction{queueDeleteMin}{deleteMin}
		\SetKwFunction{queueInsert}{insert}
		\SetKwFunction{queueMin}{min}
		\SetKwFunction{queueMinKey}{minKey}
		\SetKwFunction{queueDecreaseKey}{decreaseKey}
		\SetKwFunction{queueContains}{contains}
		\SetKwFunction{listInsert}{insert}
		\SetKwFunction{removeDom}{removeDominated}

		\SetKwProg{Pn}{Procedure}{:}{\KwRet}
		\Pn{\relaxEdge{(u,v), l}}{
			\If{$d_1(l) + \omega(u,v) < r_d$}
			{
				\D.\queueInsert{$(d_0(l) + \omega(u,v), d_1(l) + \omega(u,v), l)$}\;

				\If{$v \in P$}
				{
					\D.\queueInsert{$(d_0(l) + \omega(u,v) + d_p, 0, l)$}\;
				}

				\ForAll{ $l' \in D$ }
				{
					\If{$l'$ is not dominated by any label in \L{$v$}}
					{
						\L{$v$}.\removeDom{$l^{'}$} \;
						\L{$v$}.\queueInsert{$l^{'}$} \;
						\Q.\queueInsert{$pot(l')$}
					}
				}
			}
		}
	\end{algorithm*}
	\setlength{\interspacetitleruled}{2pt}%
	\setlength{\algotitleheightrule}{\algotitleheightruledefault}%

	\caption{\label{alg:relax_edge_a_star} Relaxing an edge with regard to the potential.}
\end{figure}

\subsection{Potential for Driving Time Constraints}
Given a target node $t$, the CH potential $\pi_{t,ch}$ yields a perfect estimate for the distance $d_{direct}(v,t)$ from $v$ to $t$ without regard for driving time restrictions and pauses. A lower bound for the time $d(v,t)$ from $v$ to $t$ with breaks due to the driving time limit can be calculated by taking the minimum necessary amount of breaks on the shortest path into account:

\[\pi{'}_t(v) = \floor*{ \frac{d_{direct}(v,t)}{t_d} } * t_p + d_{direct}(v,t)\]

A node potential is called \emph{feasible} if it does not overestimate the distance of any edge in the graph, i.e.
\begin{align}
	\label{eq:node_potential_feasibility}
	len(u,v) - pot(u) + pot(v) \ge 0 \quad \forall (u,v) \in E
\end{align}
Following example of a query using the graph in Fig. \ref{fig:graph_infeasible_potential} shows that $\pi{'}_t$ is not feasible. With a driving time limit of 6 and a pause time of 1, the potential here will yield a value $\pi_t(s) = 8$ since the potential includes the minimum required pause time for a path from s to t. Consequently, with $\pi_t(v) = 5$ and $len(s,v) = 2$, $len(s,v) - \pi_t(s) + \pi_t(v) = -1$.

\begin{figure}[hbtp]
	\centering
	\tikzstyle{node}=[circle,inner sep=0.5mm,minimum size=5.25mm,draw = black]
\tikzstyle{bright}=[fill=black!14]
\tikzstyle{dark}=[fill=black!28]
\tikzstyle{lightEdgeStyle}=[black!20]

\begin{tikzpicture}[scale=1.5, bend angle = 20]

	% Obere Reihe
	\node(s) at (1,1) [node, bright] {s};
	\node(v) at (2,1) [node, bright] {v};
	\node(t) at (3,1) [node, bright] {t};

	\draw[->] (s) -> node[midway, above]{2} (v);
	\draw[->] (v) -> node[midway, above]{5} (t);
	% \foreach \i [evaluate = \i as \lastNode using \i-1] in {2,3,...,\numberOfNodes}
	% 	{
	% 		\node (Top\i) at (\i,1) [node, bright] {\i}
	% 		edge[<-] (Top\lastNode);
	% 	}


	% % Pfeile nach rechts
	% \pgfmathparse{\numberOfNodes - 2}
	% \foreach \i [evaluate = \i as \nextNode using \i+2] in {1,2,...,\pgfmathresult}
	% 	{
	% 		\foreach \j [count=\nodeIndex from \nextNode] in {\nextNode,...,\numberOfNodes}
	% 			{
	% 				\draw[->, lightEdgeStyle] (Bot\i) to [bend right] (Bot\nodeIndex);
	% 			}
	% 	}

\end{tikzpicture}

	\caption{A graph with the potential to break the potential.}
	\label{fig:graph_infeasible_potential}
\end{figure}

A variant of the potential accounts for the distance $d(p,v)$ with $p$ being the last parking node that was used for a pause to calculate the minimum required pause time on the $v$-$t$ path. Since the potential now uses information from a label $l$ with $l \in L(v)$, it no longer is a node potential but also depends on the chosen label at $v$.

\begin{align*}
	\pi_t(l,v) & = \floor*{ \frac{d_{direct}(p,v) + d_{direct}(v,t)}{t_d} } * t_p + d_{direct}(v,t) \\
	           & = \floor*{ \frac{d_1(l) + d_{direct}(v,t)}{t_d} } * t_p + d_{direct}(v,t)
\end{align*}

Since the potential $\pi_t$ now uses label information it no longer is a node potential and the feasibility definition as defined in inequality \ref{eq:node_potential_feasibility} can no longer be applied. We still want to use potential and label information to calculate lower bound estimates for the length of paths.

\begin{lemma}
	Let $p = \langle s=v_0,v_1,...,t=v_k, \rangle$ be a path with labels $l_i$ at nodes $v_i$. Then $d_0(l_{i-1}) + \pi_t(l_{i-1},v_{i-1}) \le d_0(l_i) + \pi_t(l_i,v_i)$.
\end{lemma}

The lower bound estimate for the length of the entire path to which a label belongs can only increase when propagating labels to a next node.

\begin{proof}
	Given a Graph $G=(V,E)$ with a set of parking nodes $P \subseteq V$, let $p = \langle s=v_0,v_1,...,t=v_k, \rangle$ be a path in G with labels $l_i$ at nodes $v_i$. Let $p,q \in P \cup \{s\}$ the last parking node which was used by label $l_{i-1}$ and $l_{i}$ or $s$, if no parking node was used.

	\begin{align}
		\begin{split}\label{eq:label_feasibility_proof_1}
			d_0(l_{i-1}) + \pi_t(l_{i-1},v_{i-1}) &= d_0(l_{i-1})+\floor*{ \frac{d_1(l_{i-1}) + d_{direct}(v_{i-1},t)}{t_d} } * t_p + d_{direct}(v_{i-1},t)\\
			&= d_0(l_{i-1})+ \floor*{ \frac{d_{direct}(p,v_{i-1}) + d_{direct}(v_{i-1},t)}{t_d} } * t_p + d_{direct}(v_{i-1},t)\\
			&=d(s,p) + d_{direct}(p,v_{i-1})\\
			& \phantom{{}=1} + \underbrace{\floor*{ \frac{d_{direct}(p,v_{i-1}) + d_{direct}(v_{i-1},t)}{t_d} } * t_p}_\text{minimum required pause time on p-t subpath} + d_{direct}(v_{i-1},t)
		\end{split}
	\end{align}

	\emph{Case 1: $p=q$}

	\begin{align}
		\begin{split}\label{eq:label_feasibility_proof_2}
			d_{direct}(p,v_{i-1}) + d_{direct}(v_{i-1},t) & = d_{direct}(p,v_{i-1}) + len(v_{i-1},v_i) + d_{direct}(v_i,t) \\
			& = d_{direct}(q,v_{i-1}) + len(v_{i-1},v_i) + d_{direct}(v_i,t) \\
			& = d_{direct}(q,v_i) + d_{direct}(v_i,t)
		\end{split}
	\end{align}

	With equations \ref{eq:label_feasibility_proof_1} follows

	\begin{align}
		\begin{split}\label{eq:label_feasibility_proof_3}
			d_0(l_{i-1}) + \pi_t(l_{i-1},v_{i-1}) &= d(s,p) + d_{direct}(p,v_{i-1}) + d_{direct}(v_{i-1},t)\\
			& \phantom{{}=1} + \floor*{ \frac{d_{direct}(p,v_{i-1}) + d_{direct}(v_{i-1},t)}{t_d} } * t_p\\
			&= d(s,q) + d_{direct}(q,v_{i}) + d_{direct}(v_{i},t)\\
			& \phantom{{}=1} + \floor*{ \frac{d_{direct}(q,v_{i1}) + d_{direct}(v_{i},t)}{t_d} } * t_p\\
			&= d_0(l_{i}) + \pi_t(l_{i},v_{i})
		\end{split}
	\end{align}

	\emph{Case 2: $p \neq q$}. In this case, $q = v_i$ and $d(p,v_i) = d(p,q) = d_{direct}(p,v_i) + t_p = $. With \ref{eq:label_feasibility_proof_1} follows

	\begin{align}
		\begin{split}\label{eq:label_feasibility_proof_4}
			d_0(l_{i-1}) + \pi_t(l_{i-1},v_{i-1}) &= d(s,p) + d_{direct}(p,v_{i-1}) + d_{direct}(v_{i-1},t)\\
			& \phantom{{}=1} + \floor*{ \frac{d_{direct}(p,v_{i-1}) + d_{direct}(v_{i-1},t)}{t_d} } * t_p\\
			&= d(s,p) + d_{direct}(p,v_{i}) + d_{direct}(v_{i},t)\\
			& \phantom{{}=1} + \floor*{ \frac{d_{direct}(p,v_{i}) + d_{direct}(v_{i},t)}{t_d} } * t_p\\
			&\le d(s,p) + d(p,q) - t_p + d_{direct}(v_{i},t)\\
			& \phantom{{}=1} + \floor*{ \frac{d_{direct}(v_{i},t)}{t_d} } * t_p + t_p\\
			&= d(s,q) + 0 + d_{direct}(v_{i},t)\\
			& \phantom{{}=1} + \floor*{ \frac{0 + d_{direct}(v_{i},t)}{t_d} } * t_p\\
			&= d(s,q) + d_{direct}(q,v_{i}) + d_{direct}(v_{i},t)\\
			& \phantom{{}=1} + \floor*{ \frac{d_{direct}(q,v_{i}) + d_{direct}(v_{i},t)}{t_d} } * t_p\\
			&= d_0(l_{i}) + \pi_t(l_{i1},v_{i})
		\end{split}
	\end{align}
\end{proof}

\begin{lemma}
	The potential $\pi_t(l,v)$ of a label $l$ at a node $v$ is a lower bound for the distance including pauses from $v$ to $t$.
\end{lemma}

\begin{proof}
	Let $p = \langle s=v_0,v_1,...,t=v_k, \rangle$ be a path with labels $l_i$ at nodes $v_i$. With $d_0(l_{i-1}) + \pi_t(l_{i-1},v_{i-1}) \ge d_0(l_{i}) + \pi_t(l_i,v_i)$ for all edges on $p$, the total length $len(p)$ of the path must follow $\pi_t(l_i,v_i) \le len(p) + \pi_t(l_k,t) \Leftrightarrow l(p) \ge \pi_t(l_i,v_i) - \pi_t(l_k,t)$. Since $\pi_t(l_k,t) = 0$, $l(p) \ge \pi_t(l_i,v_i)$ holds.
\end{proof}

\begin{theorem}
	The search can be stopped when the first label at $t$ is removed from the queue.
\end{theorem}

\begin{proof}
	When a label $l$ at $t$ is removed from the queue during a $s$-$t$ query, all remaining label $m$ of a node $v$ in the queue fulfill $d_0(t) + \pi_t(l,t) \le d_0(v) + \pi_t(m,v)$. Assume that $d_0(t)$ is not the shortest distance from $s$ to $t$. Then, a shorter path $p = \langle s=v_0,v_1,...,t=v_k, \rangle$ exists which uses at least one unsettled label $m \in L(v_i)$. Since $l$ was already removed from the queue, $d_0(t) = d_0(t) + \pi_t(l,t) \le  d_0(v) + \pi_t(m,v) \le l(p)$ which contradicts the assumption that $p$ yields a shorter $s$-$t$ distance than $d_0(t)$.
\end{proof}

% \begin{proof}
% 	Given a Graph $G=(V,E)$ and any path $p = \langle v_0,v_1,...,v_k, \rangle$.
% \end{proof}
% The conditions in which $\pi_t(v)$ is feasible can be derived directly from the requirement for a feasible potential, i.e. $len(u,v) - pot(u) + pot(v) \ge 0$:

% \begin{align*}
% 	len(u,v) - \pi_t(u) + \pi_t  (v)                                                                                                                                                                                         \\
% 	 & =  len(u,v) - \left(\floor*{ \frac{d(p,u) + d_{direct}(u,t)}{t_d} } * t_p + d_{direct}(u,t)\right)                                                                                                                    \\
% 	 & \phantom{{}=1} +  \floor*{ \frac{d(p,v) + d_{direct}(v,t)}{t_d} } * t_p + d_{direct}(v,t)                                                                                                                             \\
% 	 & =  \floor*{ \frac{d(p,v) + d_{direct}(v,t)}{t_d} } * t_p - \floor*{ \frac{d(p,u) + d_{direct}(u,t)}{t_d} } * t_p                                                                                                      \\
% 	 & \phantom{{}=1} + len(u,v) + d_{direct}(v,t) - d_{direct}(u,t)                                                                                                                                                         \\
% 	 & =  \underbrace{ \floor*{ \frac{d(p,v) + d_{direct}(v,t)}{t_d} } * t_p}_\text{min. pause time on s-v-t path} - \underbrace{\floor*{ \frac{d(p,u) + d_{direct}(u,t)}{t_d} } * t_p}_\text{min. pause time on s-u-t path}
% \end{align*}

% The potential therefore is infeasible if the minimum required pause time on an $s$-$t$ path via $u$ is greater than the minimum required pause time on an $s$-$t$ path via $v$. An example graph where this case occurs is given in Fig. \ref{fig:graph_infeasible_potential_2}.
% \begin{figure}[hbtp]
% 	\centering
% 	\tikzstyle{node}=[circle,inner sep=0.5mm,minimum size=5.25mm,draw = black]
\tikzstyle{bright}=[fill=black!14]
\tikzstyle{dark}=[fill=black!28]
\tikzstyle{lightEdgeStyle}=[black!20]

\begin{tikzpicture}[scale=1.5, bend angle = 20]

	% Obere Reihe
	\node(u) at (2,3) [node, bright] {u};
	\node(s) at (1,2) [node, bright] {s};
	\node(t) at (3,2) [node, bright] {t};
	\node(v) at (2,1) [node, bright] {v};

	\draw[->] (s) -> node[midway, above]{2} (u);
	\draw[->] (u) -> node[midway, above]{2} (t);
	\draw[->] (s) -> node[midway, above]{1} (v);
	\draw[->] (v) -> node[midway, above]{1} (t);
	\draw[->] (u) -> node[midway, right]{1} (v);
	% \foreach \i [evaluate = \i as \lastNode using \i-1] in {2,3,...,\numberOfNodes}
	% 	{
	% 		\node (Top\i) at (\i,1) [node, bright] {\i}
	% 		edge[<-] (Top\lastNode);
	% 	}


	% % Pfeile nach rechts
	% \pgfmathparse{\numberOfNodes - 2}
	% \foreach \i [evaluate = \i as \nextNode using \i+2] in {1,2,...,\pgfmathresult}
	% 	{
	% 		\foreach \j [count=\nodeIndex from \nextNode] in {\nextNode,...,\numberOfNodes}
	% 			{
	% 				\draw[->, lightEdgeStyle] (Bot\i) to [bend right] (Bot\nodeIndex);
	% 			}
	% 	}

\end{tikzpicture}

% 	\caption{A graph with the potential to break the potential.}
% 	\label{fig:graph_infeasible_potential_2}
% \end{figure}

% With a driving time limit of 3 and a pause time of 1, an $s$-$t$ path via $u$ needs a pause time of 1 while an $s$-$t$ path via $v$ does not pause. Therefore, $len(u,v) - \pi_t(u) + \pi_t(v) = -1$ and the feasibility condition does not hold.


\section{Multiple Driving Time Constraints}
\section{Core Contraction Hierarchy Variant}
\begin{algorithm}[hbtp]
	\caption{\textsc{Core-CH with Driving Time Constraints}}\label{alg:CSPCoreCH}

	% Some settings
	\DontPrintSemicolon %dontprintsemicolon
	\SetFuncSty{textsc}
	\SetKwFor{ForAll}{forall}{do}

	% Declaration of data containers and functions
	\SetKwData{Q}{Q}
	\SetKwData{L}{L}
	\SetKwData{pot}{pot}
	\SetKwFunction{queueDeleteMin}{deleteMin}
	\SetKwFunction{queueInsert}{insert}
	\SetKwFunction{queueMin}{min}
	\SetKwFunction{queueMinKey}{minKey}
	\SetKwFunction{queueDecreaseKey}{decreaseKey}
	\SetKwFunction{queueContains}{contains}
	\SetKwFunction{listInsert}{insert}
	\SetKwFunction{removeDom}{removeDominated}

	% Algorithm interface
	\KwIn{Graph $G = (V,E,\omega)$, parking nodes $P \subseteq V$, driving time restriction $r$, potential \pot{}, source node $s \in V$}
	\KwData{Priority queue \Q, per node priority queue \L{$v$} of labels for all $v \in V$}
	\KwOut{Distances for all $v \in V$, tree of allowed shortest paths according to the restriction $r$ from $s$, given by $l_{pred}$}

	% The algorithm
	\BlankLine
	\tcp{Initialization}
	\Q.\queueInsert{$s,(0,0)$}\;
	\L{$s$}.\queueInsert{$(\bot,\bot),\pot{(0,0)}$}\;
	\BlankLine
	\tcp{Main loop}
	\While{\Q is not empty}
	{
		$u \leftarrow$ \Q.\queueDeleteMin{} \;
		$(d_0, d_1) \leftarrow$ \L{$u$}.\queueMinKey{} \;
		$l \leftarrow$ \L{$u$}.\queueDeleteMin{} \;
		\BlankLine
		\If{\L{$u$} is not empty}
		{
			$k_{dist} \leftarrow$ \L{$u$}.\queueMinKey{} \;
			\Q.\queueInsert{$u$, $k_{dist}$} \;
		}

		\ForAll{ $(u,v) \in E$ }
		{
			\If{$d_0 + \omega(u,v) < r_d$}
			{
				$D \leftarrow \{(d_0 + \omega(u,v), d_1 + \omega(u,v))\}$

				\If{$v \in P$}
				{
					$D$.\listInsert{$(d_0 + \omega(u,v) + r_p, 0)$}
				}

				\ForAll{ $x \in D$ }
				{
					\If{$x$ is not dominated by any label in \L{$v$}}
					{
						\L{$v$}.\removeDom{$x$} \;
						\L{$v$}.\queueInsert{$(l,(u,v)), x$} \;
						\uIf{\Q.\queueContains{v}}
						{
							\Q.\queueDecreaseKey{$v, x$}
						}
						\Else
						{
							\Q.\queueInsert{$v, x$}
						}
					}
				}
			}
		}
	}
\end{algorithm}
\subsection{Building the Contraction Hierarchy}

\section{Combining A* and Core Contraction Hierarchy}
% !TeX root = thesis.tex
%% evaluation.tex
%%

%% ==============
\chapter{Evaluation\label{ch:Evaluation}}
\section{Theoretical Complexity}
TODO is multi-criteria spp with travel time and the time since last break as criteria, optimal multi-criteria path is not prefix optimal  \cite{tuin:2018}

according to \cite{hansen:1980} np-hard

\section{Experiments}
In this Section, we evaluate the running time and behavior of our algorithms of chapter~\ref{ch:Algorithm} in various experiments and describe the underlying data. Our machine runs openSUSE Leap 15.3, has \SI{128}{\giga\byte} (8x\SI{16}{\giga\byte}) of \SI{2133}{\mega\hertz} DDR4 RAM, and a 4-core Intel Xeon E5-1630v3 CPU which runs at \SI{3.7}{\giga\hertz}. The code is written in Rust and compiled with cargo 1.59.0-nightly using the release profile with lto~=~true and codegen-units~=~1.

\subparagraph{Data.} Our data is a road network of Europe from Open Street Map\footnote{\url{https://download.geofabrik.de/europe-latest.osm.pbf} of March 22, 2022} (OSM). We extract the routing graph from the OSM data using RoutingKit\footnote{\url{https://github.com/RoutingKit/RoutingKit}}, respectively a custom extension\footnote{\url{https://github.com/maxoe/RoutingKit}} of RoutingKit which is capable of extracting parking nodes accordingly and building the core CH as described in Section \ref{sec:build_corech}. The obtained routing graph has $81.5$ million nodes and $190$ million edges. If not stated otherwise, our set $P$ of parking nodes consists of 6800 nodes which were selected due to their OSM attributes. We will later conduct experiments with different choices for $P$.

\subparagraph{Methodology.} All experiments are run sequentially. We conduct experiments regarding the preprocessing time of the OSM data and the running time of queries on the extracted routing graph. We average preprocessing running times over \num{10} runs and running times of $s$-$t$ queries over \num{10000} queries with $s$ and $t$ independently chosen uniformly at random for each query. If not stated otherwise, we use $\restr_1$ with $\restr_1^d = \SI{4.5}{\hour}$ and $\restr_1^b = \SI{0.45}{\minute}$ and $\restr_2$ with $\restr_2^d = \SI{9}{\hour}$ and $\restr_2^b = \SI{9}{\hour}$ to approximate the regulations of the EU.

\subsection{Comparison of Algorithms and Optimizations}
We evaluate the different algorithms of chapter \ref{ch:Algorithm} and the optimizations of chapter \refeq{ch:impl}.

First, we compare the running times of queries of the algorithms of chapter \ref{ch:Algorithm} on a German road network which was obtained in the exact same way from OSM as the European road network\footnote{\url{https://download.geofabrik.de/europe/germany-latest.osm.pbf} of March 22, 2022}. We scale the experiment down from the European road network because some variants of the algorithms in this experiment cannot keep up with the performance of the goal-directed core CH algorithm and would render the experiment slow and impracticable. To avoid distortions, we also scale down the driving time constraints. Otherwise, most random $s$-$t$ queries on the German road network would be to short to require a break and almost no query would require more than one break. TODO

As Table \ref{tbl:extensions_runtime} shows, the goal-directed search already performs an order of magnitude better than the baseline Dijkstra's algorithm with our amendments for driving time constraints. TODO bidir. The largest improvement brings the bidirectional search which improves the baseline by a factor of about \num{10000} and the goal-directed variant by a factor of about \num{1000}.

\begin{table}[hbtp]
	\centering
	\begin{tabular}{cccrrrr}
	\toprule
	              &               &         & \multicolumn{2}{c}{Running Time [\si{\milli\second}]}            \\
	Goal-Directed & Bidirectional & Core CH & 1-DTC                                                 & 2-DTC    \\
	\midrule
	\xmark        & \xmark        & \xmark  & 35110.14                                              & 35110.14 \\
	\cmark        & \xmark        & \xmark  & 1393.72                                               & 1393.72  \\
	\xmark        & \cmark        & \xmark  & 50463.27                                              & 50463.27 \\
	\cmark        & \cmark        & \xmark  & 4.38                                                  & 4.38     \\
	\xmark        & \cmark        & \cmark  & 121.29                                                & 121.29   \\
	\cmark        & \cmark        & \cmark  & 3.47                                                  & 3.47     \\
	\bottomrule
\end{tabular}
	\caption{Running times of random queries on a German road network}
	\label{tbl:extensions_runtime}
\end{table}

Most of the performance gain of the goal-directed search originates from the very tight lower-bound given by the CH potentials. If the shortest travel time between two nodes $s$ and $t$ is way larger than $\distance(s,t)$ due to necessary breaks and even detours to parking nodes on the route, then the performance of the goal-directed search degrades. The bidirectional variant can mitigate this disadvantage since it connects two routes which each for itself have fewer breaks on the route. A disadvantage from the goal-directed search which cannot be mitigated by the bidirectional variant are its outliers. In cases where the algorithm is not able to find a route or the route needs a lot of breaks, the running time increases significantly. TODO

TODO outlier problem with graphics leas to non ch excluded even if fast
\begin{table}[hbtp]
	\centering
	\begin{tabular}{cccrrrr}
	\toprule
	              &               &         & \multicolumn{2}{c}{Running Time [\si{\milli\second}]}         \\
	Goal-Directed & Bidirectional & Core CH & 1-DTC                                                 & 2-DTC \\
	\midrule
	\xmark        & \cmark        & \cmark  & 316.88                                                     & 316.88     \\
	\cmark        & \cmark        & \cmark  & 118.32                                                     & 118.32     \\
	\bottomrule
\end{tabular}
	\caption{Running times of random queries on a European road network}
	\label{tbl:extensions_runtime_eur}
\end{table}

Second, we evaluate the different optimizations as described in chapter \refeq{ch:impl} in use with the goal-directed core CH algorithm on the European road network.

\begin{table}[hbtp]
	\centering
	\begin{tabular}{crrrr}
	\toprule
	                 & \multicolumn{2}{c}{Mean [\si{\milli\second}]} & \multicolumn{2}{c}{Median [\si{\milli\second}]}                 \\
	Backward Pruning & 1-DTC                                         & 2-DTC                                           & 1-DTC & 2-DTC \\
	\midrule
	\xmark           & 131.89                                             & 262.37                                               & 45.04     & 27.10     \\
	\cmark           & 115.20                                             & 235.18                                               & 36.14     & 24.64     \\
	\bottomrule
\end{tabular}

	\caption{Comparison of running times of the goal-directed core CH algorithm with different optimizations from Section\ref{ch:impl}.}
	\label{tbl:opt_runtime}
\end{table}

\subsection{Study of Goal-Directed Core CH Queries}
Finally, we investigate queries of the goal-directed core CH algorithm with full optimizations more closely. What drives the running time of the algorithm? TODO

\begin{figure}[hbtp]
	\centering
	\includegraphics[width=.95\textwidth]{plots/measure_all_csp_2_1000_queries_rank_times-core_ch_chpot-time_ms.png}
	\caption{Running times of the goal-directed core CH algorithm for queries to nodes of increasing Dijkstra rank, logarithmic scales.}
	\label{fig:rank_times}
\end{figure}

\begin{itemize}
	\item different $\len$
	\item variable break time, driving time limit
	\item parking node set
\end{itemize}

%% conclusion.tex
%%

%% ==================
\chapter{Conclusion}
\label{ch:conclusion}
%% ==================
conclusion
- motivtion recap: long-haul
- abstraction tdrp
- algorithm baseline and goal-directed and core ch extensions
- evaluation shows ... results


outlook
- theoretical analysis of n dtc
- exact model of dtc
- merging with time-dependent for road closures and driving bans





%% --------------------
%% |   Bibliography   |
%% --------------------

\cleardoublepage
\phantomsection
\addcontentsline{toc}{chapter}{\bibname}

\iflanguage{english}
{\bibliographystyle{alpha}}
{\bibliographystyle{babalpha-fl}} % german style

\bibliography{references}


%% ----------------
%% |   Appendix   |
%% ----------------

\cleardoublepage
% !TeX root = thesis.tex
%% appendix.tex
%%

%% ==============================
%\chapter{Appendix}
%\label{ch:Appendix}
%% ==============================

\appendix

\iflanguage{english}
{\addchap{Appendix}}	% english style
{\addchap{Anhang}}	% german style

\section{List of Abbreviations}
\label{app:abb}

\begin{tabular}{rp{0.85\textwidth}}
	SPP   & Shortest Path Problem                                                                         \\
	OSM   & Open Street Map                                                                               \\
	1-DTC & Restriction of the long-haul truck driver routing problem to only one driving time constraint \\
	CH    & Contraction Hierarchy                                                                         \\
\end{tabular}



\section{List of Symbols and Designations}
\label{app:symbols}

We use Greek symbols for theoretical and abstract concepts such as the shortest distance between two nodes or an arbitrary valid node potential. Concrete functions and procedures, i.e., those with definitions, have verbose names. Algorithmic functions and procedures for which pseudocode is provided use \algofunction{SmallCaps}.

\begin{tabular}{rp{0.85\textwidth}}
	$\potential_t(v)$         & A node potential to $t$                                                                                                                   \\
	$\distance(s,t)$          & Shortest distance between $s$ and $t$.                                                                                                    \\
	$\traveltime(s,t)$        & Shortest travel time between $s$ and $t$.                                                                                                 \\
	$\drivingtime(s,t)$       & Driving time of the shortest route between $s$ and $t$                                                                                    \\
	\\
	$\route$                  & A route consisting of a path and a function $\breakTime(v)$ function.                                                                     \\
	$\restr$                  & A driving time constraint consisting of maximum allowed driving time $c^d$ and break time $c^b$                                           \\
	$\restrset$               & A set of one or more driving time constraints                                                                                             \\
	$L(v)$                    & The label set of a node $v$                                                                                                               \\
	$\breakDist_i(l)$         & Distance since the last break of a label with a duration of at least $c_i^b$ or distance since the last break if the index $i$ is omitted \\
	$\pred(l)$                & Predecessor label of a label $l$.                                                                                                         \\

	$\concretedt(l)$          & Driving time of a label $l$                                                                                                               \\
	$\concretett(l)$          & Travel time of a label $l$                                                                                                                \\
	$\concretepotential(l,v)$ & The potential of a label $l \in L(v)$ as defined in section \ref{section:potential_n_csp}                                                 \\
	$\len((u,v))$             & The weight, respectively length of an edge $(u,v)$, alternatively used for the length $\len(p)$ of a path $p$                             \\
\end{tabular}






\end{document}
