\documentclass{thesisclass}
% Based on thesisclass.cls of Timo Rohrberg, 2009
% ----------------------------------------------------------------
% Thesis - Main document
% ----------------------------------------------------------------


%% -------------------------------
%% |  Information for PDF file   |
%% -------------------------------
\hypersetup{
 pdfauthor={Max Oesterle},
 pdftitle={Efficient Long-Haul Truck Driver Routing},
 pdfsubject={?},
 pdfkeywords={?}
}


%% ---------------------------------
%% | Information about the thesis  |
%% ---------------------------------

\newcommand{\myname}{Max Oesterle}
\newcommand{\mytitle}{Efficient Long-Haul Truck Driver Routing}
\newcommand{\myinstitute}{Institute of Theoretical Informatics}

\newcommand{\reviewerone}{Dr. rer. nat. Torsten Ueckerdt}
\newcommand{\reviewertwo}{?}
\newcommand{\advisor}{Tim Zeitz}
\newcommand{\advisortwo}{Alexander Kleff}
\newcommand{\advisorthree}{Frank Schulz}

\newcommand{\timestart}{15th January 2022}
\newcommand{\timeend}{15th July 2022}


%% ---------------------------------
%% | Commands                      |
%% ---------------------------------

\newtheorem{definition}{Definition} \numberwithin{definition}{chapter}
\newtheorem{theorem}[definition]{Theorem}
\newtheorem{lemma}[definition]{Lemma}
\newtheorem{corollary}[definition]{Corollary}
\newtheorem{conjecture}[definition]{Conjecture}

%% --------------------------------
%% | Settings for word separation |
%% --------------------------------
% Help for separation:
% In german package the following hints are additionally available:
% "- = Additional separation
% "| = Suppress ligation and possible separation (e.g. Schaf"|fell)
% "~ = Hyphenation without separation (e.g. bergauf und "~ab)
% "= = Hyphenation with separation before and after
% "" = Separation without a hyphenation (e.g. und/""oder)

% Describe separation hints here:
\hyphenation{
% Pro-to-koll-in-stan-zen
% Ma-na-ge-ment  Netz-werk-ele-men-ten
% Netz-werk Netz-werk-re-ser-vie-rung
% Netz-werk-adap-ter Fein-ju-stier-ung
% Da-ten-strom-spe-zi-fi-ka-tion Pa-ket-rumpf
% Kon-troll-in-stanz
}


%% ------------------------
%% |    Including files   |
%% ------------------------
% Only files listed here will be included!
% Userful command for partially translating the document (for bug-fixing e.g.)
\includeonly{
titlepage,
introduction,
preliminaries,
problem,
algorithm,
implementation,
evaluation,
conclusion,
appendix
}

\usepackage{mathtools}
\DeclarePairedDelimiter{\floor}{\lfloor}{\rfloor}
\DeclarePairedDelimiter{\ceil}{\lceil}{\rceil}

% styles
% \newcommand{\mathfunction}[1]{\ensuremath{\boldsymbol{#1}}}
% \newcommand{\mathfunction}[1]{\texttt{\itshape #1}}
\newcommand{\mathfunction}[1]{\ensuremath{\operatorname{#1}}}
\newcommand{\algofunction}[1]{\textsc{#1}}

% symbols and names

% theoretical
\newcommand{\potential}{\pi}
\newcommand{\distance}{\mu}
\newcommand{\traveltime}{\mu_{tt}}
\newcommand{\drivingtime}{\mu_{dt}}

% concrete functions
\newcommand{\len}{\mathfunction{len}}
\newcommand{\concretepotential}{\mathfunction{pot}}
\newcommand{\concretett}{\mathfunction{travelTime}}
\newcommand{\concretedt}{\mathfunction{drivingTime}}
\newcommand{\breakDist}{\mathfunction{breakDist}}
\newcommand{\pred}{\mathfunction{pred}}
\newcommand{\chpotential}{\mathfunction{chPot}}
\newcommand{\breakTime}{\mathfunction{breakTime}}
\newcommand{\minBreaks}{\mathfunction{minBreaks}}
\newcommand{\breakEstimate}{\mathfunction{breakEstimate}}
\newcommand{\minBreakTime}{\mathfunction{minBreakTime}}

% data
\newcommand{\tenttraveltime}{\ensuremath{tent}}
\newcommand{\restr}{\ensuremath{c}}
\newcommand{\restrset}{\ensuremath{C}}
\newcommand{\route}{\ensuremath{r}}


%%%%%%%%%%%%%%%%%%%%%%%%%%%%%%%%%
%% Here, main documents begins %%
%%%%%%%%%%%%%%%%%%%%%%%%%%%%%%%%%
\begin{document}

% Remove the following line for German text
\selectlanguage{english}

\frontmatter
\pagenumbering{roman}
%% titlepage.tex
%%

% coordinates for the bg shape on the titlepage
\newcommand{\diameter}{7}
\newcommand{\xone}{-15}
\newcommand{\xtwo}{160}
\newcommand{\yone}{15}
\newcommand{\ytwo}{-253}

\begin{titlepage}
	% bg shape
	\begin{tikzpicture}[overlay]
		\draw[color=gray]
		(\xone mm, \yone mm)
		-- (\xtwo mm, \yone mm)
		arc (90:0:\diameter mm)
		-- (\xtwo mm + \diameter mm , \ytwo mm)
		-- (\xone mm + \diameter mm , \ytwo mm)
		arc (270:180:\diameter mm)
		-- (\xone mm, \yone mm);
	\end{tikzpicture}
	\begin{textblock}{10}[0,0](4,2.5)
		\includegraphics[width=.3\textwidth]{logos/KITLogo.pdf}
	\end{textblock}
	\begin{textblock}{10}[0,0](14.5,2.45)
		\includegraphics[width=.15\textwidth]{logos/algoLogo.pdf}
	\end{textblock}
	\changefont{phv}{m}{n}	% helvetica	
	\vspace*{3.75cm}
	\begin{center}
		\Huge{\mytitle}
		\vspace*{2.25cm}\\
		\Large{
			\iflanguage{english}{Master Thesis of}
			{Diplomarbeit\\von}
		}\\
		\vspace*{1cm}
		\huge{\myname}\\
		\vspace*{1cm}
		\Large{
			\iflanguage{english}{At the Department of Informatics}
			{An der Fakult\"at f\"ur Informatik}
			\\
			\myinstitute
		}
	\end{center}
	\vspace*{1cm}
	\Large{
		\begin{center}
			\begin{tabular}[ht]{l c l}
				% Gutachter sind die Professoren, die die Arbeit bewerten. 
				\iflanguage{english}{Reviewers}{Erstgutachter}:         & \hfill & \reviewerone  \\
				\iflanguage{english}{}{Zweitgutachter:}                 & \hfill & \reviewertwo  \\
				\iflanguage{english}{Advisors}{Betreuende Mitarbeiter}: & \hfill & \advisor      \\
				\iflanguage{english}{}{}                                & \hfill & \advisortwo   \\
				\iflanguage{english}{}{}                                & \hfill & \advisorthree \\
				% Der zweite betreuende Mitarbeiter kann weggelassen werden. 
			\end{tabular}
		\end{center}
	}


	\vspace{2cm}
	\begin{center}
		\large{\iflanguage{english}{Time Period}{Bearbeitungszeit}: \ \timestart{} \ -- \ \timeend}
	\end{center}


	\begin{textblock}{10}[0,0](4,16.8)
		\tiny{
			\iflanguage{english}
			{KIT -- The Research University in the Helmholtz Association}
			{KIT -- Die Forschungsuniversit\"at in der Helmholtz-Gemeinschaft}
		}
	\end{textblock}

	\begin{textblock}{10}[0,0](14,16.75)
		\large{
			\textbf{www.kit.edu}
		}
	\end{textblock}

\end{titlepage}

\blankpage

%% -------------------------------
%% |   Statement of Authorship   |
%% -------------------------------

\thispagestyle{plain}

\vspace*{\fill}

\centerline{\textbf{Statement of Authorship}}

\vspace{0.25cm}

Ich versichere wahrheitsgemäß, die Arbeit selbstständig verfasst, alle benutzten Hilfsmittel vollständig und genau angegeben und alles kenntlich gemacht zu haben, was aus Arbeiten anderer unverändert oder mit Abänderungen entnommen wurde sowie die Satzung des KIT zur Sicherung guter wissenschaftlicher Praxis in der jeweils gültigen Fassung beachtet zu haben.

\vspace{2.5cm}

\hspace{0.25cm} Karlsruhe, \today

\vspace{2cm}

\blankpage

%% -------------------
%% |   Abstract      |
%% -------------------

\thispagestyle{plain}

\begin{addmargin}{0.5cm}

	\centerline{\textbf{Abstract}}

	A short summary of what is going on here.

	\vskip 2cm

	\centerline{\textbf{Deutsche Zusammenfassung}}

	Kurze Inhaltsangabe auf deutsch.

\end{addmargin}

\blankpage

%% -------------------
%% |   Directories   |
%% -------------------

\tableofcontents
\blankpage

%% -----------------
%% |   Main part   |
%% -----------------

\mainmatter
\pagenumbering{arabic}
%% introduction.tex

\chapter{Introduction\label{ch:introduction}}

For many professional truck drivers, fatigue is a daily companion while en-route. Early surveys \cite{williamson:2001, adams-guppy:2003} interviewing drivers in different countries all over the world conclude that fatigue is experienced by drivers daily and within hours of the start of the drive. Supporting drivers in scheduling breaks as required can reduce fatigue. This is especially important since fatigue is a common cause for accidents and near misses. The investigation of \cite{evers:2005} observed severe truck accidents in Germany for a period of three months. They found that in $16\%$ of the 127 registered accidents that were caused by the truck driver, fatigue was mentioned as the cause of the accident. A study of causes of truck crashes in the US \cite{federalmotorcarriersafetyadministrationfmcsa:2006} examines 967 crashes during the years 2001 to 2003 and found that $13\%$ of the truck drivers stated that they experienced fatigue during the time of the crash. In \cite{connor:2001}, the authors find a strong relationship between acute fatigue and crash involvement by interviewing crash-involved and non-crash-involved drivers in the same areas and during the same times. In a similar case-control-study, \cite{cummings:2001} find a fourteen-fold increased crash risk for drivers who have experienced fatigue to a level where they almost have fallen asleep in the driver's seat. Today, fatigue is still a common cause for general accidents on the road, not only for accidents which involve trucks \cite{statistischesbundesamtdestatis:2021}, a finding that is also supported by a large meta-study \cite{moradi:2019} which additionally finds that strategies to reduce driver fatigue can effectively reduce the risk of traffic accidents.

Regulators reacted early to reduce the risk of accidents caused by fatigued truck drivers and introduced regulations, often using the term \emph{drivers' working hours} or \emph{hours of service} in the United States. In 1985, the Council of the European Communities (since 1993 the Council of the European Union) passed council regulation (EEC) No 3820/85 \cite{counciloftheeuropeancommunities:1985} which, with some exceptions, introduced a mandatory break of \SI{45}{\minute} after \SI{4.5}{\hour} of driving and a daily rest period of \SI{11}{\hour} for professional drivers ``involved in the carriage of goods''. In addition, drivers must extend the daily rest period to \si{45} consecutive hours once every week or, if exceptions apply, must compensate for a reduced weekly rest period within three weeks. Directive 2002/15/EC \cite{europeanparliament:2002} of the European Parliament and of the Council governs working times and introduced a maximum average weekly working time of \SI{48}{\hour} over four months and a maximum weekly working time of \SI{60}{\hour}. Since driving time is working time, the directive also applies for truck drivers. The regulation that determines the driving time limits, break, and rest times of today was introduced in 2006 as EC 561/2006 of the European Union \cite{europeanparliament:2006}. It contains core principles such as the break after \SI{4.5}{\hour} and the daily and weekly rest periods. Since this regulation is of greater importance of this work, we present a brief study of it in Section~TODO.

In the US, the first hours of services regulations for long-haul truck drivers were adopted in the late 1930s - a time in which trucks achieved average speeds on routes of about \SI[per-mode = symbol]{40}{\km\per\hour} \cite{federalmotorcarriersafetyadministrationfmcsa:2000}. The regulation limited working hours to \SI{12}{\hour} within a period of \SI{15}{\hour} and introduced a weekly maximum of \SI{60}{\hour}. The US HOS regulations for truck drivers of today allow a maximum of \SI{8}{\hour} of driving until the driver must take a break for at least \SI{30}{\minute} and a maximum \SI{11}{\hour} of on-duty work until the driver must rest for \SI{10}{\hour} off duty \cite{federalmotorcarriersafetyadministrationfmcsa:2011}. As for the current EU regulation, we will present a more detailed study of the US rules in Section~TODO.

Driver's working hours regulations increase the road safety and the working conditions of truck drivers. At the same, time they impose a burden on the truck driver who is responsible for following the regulations and taking the necessary breaks and rest times. Additionally, the regulations impose the challenge for the driver of finding appropriate places for parking along the route which cause as little as possible detours from the original route. To reduce the necessary manual routing effort from the truck driver, the search for parking locations for mandatory breaks ideally is incorporated into the navigation system which is used to navigate towards the destination. This necessitates routing algorithms which are capable of determining the need for breaks on a route and finding parking locations which lead to the smallest possible detours while complying with given drivers' working hours regulations.

Existing work manages to find the routes to a destination which minimize the sum of driving time and break time if the number of breaks on the route is limited to one break. The EU regulations allow routes up to a driving time of \SI{9}{\hour} with one break of \SI{45}{\minute} at the half-way point. Therefore, limiting the number of possible breaks on a route to one break is suitable for finding shorter routes with a travel time of not longer than a day, whereas long-haul truck drivers who drive multi-day routes are left out with this approach since the maximum travel time of a route which can be found with this limitation is limited. Additionally, long-haul truck drivers must consider a more complex set of regulations that mandates daily and weekly rest times, many of which are not relevant for planning short routes. We name the problem of finding a shortest route to a destination in an arbitrary distance while following a given set of driver's working hours regulations the Long-Haul Truck Driver Routing Problem (LH-TDRP). To the best of our knowledge, there exists no approach which allows an unlimited number of breaks on the route and achieves practicable performance.

In this thesis, we will begin with an overview over related literature on truck driver routing or scheduling in Chapter~\ref{ch:related_work} where we also discuss the driver's working hours regulations of the EU and the US. Chapter~\ref{ch:preliminaries} introduces our notations and the foundations on which we base our work. In Chapter~\ref{ch:problem_definitions}, we then abstract from the LH-TDRP and its variants using the aforementioned abstractions from the EU and US regulations to obtain the Truck Driver Routing Problem (TDRP), which we will define formally. In Chapter~\ref{ch:Algorithm}, we present a baseline algorithm and use the foundations of Chapter~\ref{ch:preliminaries} to develop extensions and optimizations that lead to practicable running times. Finally, we conduct a series of experiments regarding the running time of our algorithms on a German and European road network in Chapter~\ref{ch:Evaluation}. Additionally, we investigate how different parameters and road networks influences the performance of our algorithms. We conclude this work in Chapter \ref{ch:conclusion} with a short summary and suggestions on possible future research.
% !TeX root = thesis.tex
%% preliminaries.tex
%%

%% ==============
\chapter{Preliminaries\label{ch:preliminaries}}
%% ==============
In this chapter, we will introduce our basic notation and discuss important algorithmic concepts on which the work of this thesis is based.

We define a weighted, directed graph $G$ as a tuple $G=(V,E,\mathfunction{len})$. $V$ is the set of nodes and $E$ the set of edges $(u,v) \subseteq V \times V$ between those nodes. The function $\mathfunction{len}$ is the weight function $\mathfunction{len}\colon E \rightarrow \mathbb{R}_{\ge 0}$ which assigns each edge a non-negative weight which we often also call length of an edge. A path $p$ in $G$ is defined as a sequence of nodes $p = \langle v_0,v_1,...,v_k \rangle$ with $(v_i,v_{i+1}) \in E$. For simplicity, we will reuse the same function $\mathfunction{len}$ which we use to denote the length of an edge, to denote the length of a path $p$ in $G$. The length of a path $\mathfunction{len}(p)$ is defined by the sum of the weights of the edges on the path $\mathfunction{len}(p) = \sum_{i=0}^{k-1}{\mathfunction{len}((v_i,v_{i+1}))}$. A path must not necessarily be simple, i.e. nodes can appear multiple times in the same path.

Given two nodes $s$ and $t$ in a graph, we denote the shortest distance between them as $\distance(s,t)$. The shortest distance between two nodes is the minimum length of a path between them. The problem of finding the shortest distance and the associated path between two nodes in a graph is called the shortest path problem which we often abbreviate as SPP. The problem of finding the shortest path between nodes in a road network can be formalized as solving the SPP on a weighted, directed graph. Each edge of the graph represents a road and each node represents an intersection. Unless stated otherwise, the length $\len$ of an edge $(u,v)$ will always correspond to the time it takes to travel from $u$ to $v$ on the road which the edge represents. A solution of the SPP then yields the shortest time between two intersections in the road network and the associated path between them.

Dijkstra's algorithm, published in 1959, solves the SPP \cite{dijkstra:1959}. It operates on the graph G without any additionally information or precomputed data structures. It maintains a queue $Q$ of nodes with ascending tentative distance from the starting node $s$ and two arrays, a distance value $d[v]$ and a predecessor node $pred[v]$ for each node. At the beginning, $Q$ only contains the start node $s$ with the distance zero. The two arrays are initialized with $d[v]=\infty$ and $pred[v]=\bot$ except for $d[s]=0$ and $pred[s]=s$. Iteratively, the node $u$ with the minimum distance is removed from $Q$ and each outgoing edge $(u,v) \in E$ of $u$ is \emph{relaxed}. We call this process \emph{settling} a node $u$. Relaxing an edge $(u,v)$ consists of three steps: First, the sum $d[u] + \len((u,v))$ is calculated. Second, it is tested if the distance $d[v]$ can be improved by choosing $u$ as a predecessor. Finally, if that is the case, the queue key of the node $v$ is decreased. If $v$ is not contained in $Q$ yet, the node is inserted into $Q$. The search can be stopped if the target node $t$ was removed from the queue \cite{dijkstra:1959}.

For many practical applications and for large graphs, Dijkstra's algorithm is too slow. A common extension is the A* algorithm \cite{hart:1968}. A* uses a \emph{heuristic} which yields a lower bound for the distance from each node to the target node to direct the search towards the goal. With a tight heuristic, A* can significantly reduce the search space, i.e., the amount of nodes it touches during the search in comparison to Dijkstra's algorithm. In the route planning context, the heuristic often is called \emph{potential}. We will denote the potential of a node $v$ to a node $t$ with $\potential_t(v)$.

To further reduce the search space, it is possible to run a \emph{bidirectional} search. A bidirectional search to solve the SPP from $s$ to $t$ on a graph $G$ consists of a forward search and a backward search. The forward search operates on $G$ with start node $s$ and target node $t$ and maintains a forward queue $\overrightarrow{Q}$, a forward distance array $\overrightarrow{d}[v]$, and a forward predecessor array $\overrightarrow{pred}[v]$. The backward search operates on a backward graph $\overleftarrow{G}$ with start node $t$ and target node $s$ and maintains a backward queue $\overleftarrow{Q}$, a backward distance array $\overleftarrow{d}[v]$, and a backward predecessor array $\overleftarrow{pred}[v]$. The backward graph is defined as the graph $G$ with inverted edges, i.e, $\overleftarrow{G} = (V,\overleftarrow{E},\overleftarrow{\mathfunction{len}})$ with $\overleftarrow{E} = \{(v,u) \in V \times V \mid (u,v) \in E\}$ and $\overleftarrow{\mathfunction{len}}(u,v) = \mathfunction{len}(v,u)$. Additionally, an array $d[v]$ is maintained for combined tentative distances of forward and backward search and is initialized with $\infty$ for all nodes. A value $d[v]$ constitutes the shortest known distance for an $s$-$t$ path using $v$.

Forward and backward search now alternately settle a node $v$. If $v$ was settled by both searches, forward and backward search met at $v$. The value $d[v]$ is updated to the sum of $\overrightarrow{d}[v] + \overleftarrow{d}[v]$ if it is an improvement over the old value, i.e., it yields a shorter distance for an $s$-$t$ path via $v$.

The bidirectional search only yields an advantage over a unidirectional search if the two searches are stopped earlier than in a unidirectional search. If not, the bidirectional search would simply execute the work of an $s$-$t$ search twice. Therefore, stronger stopping criteria are introduced. When introducing a strong stopping criterion for a bidirectional A* search, the stopping criterion also depends on the potential function being used. The work of \cite{goldberg:2005} introduces a potential and stopping criterion which leads to an improvement over a unidirectional A* search. If a forward potential $\overrightarrow{\potential}_t$ and a backward potential $\overleftarrow{\potential}_s$ is used for the forward and the backward search, the search can be stopped when the minimum key of the forward queue $\minKey(\overrightarrow{Q})$ or the minimum key of the backward queue $\minKey(\overrightarrow{Q})$ is greater than the currently known shortest distance between $s$ and $t$. If $\overrightarrow{\potential}_t + \overleftarrow{\potential}_s \equiv const.$ holds for the potentials, then the search can be stopped when $\minKey(\overrightarrow{Q}) + \minKey(\overleftarrow{Q})$ exceeds the shortest know distance between $s$ and $t$.

\section{Contraction Hierarchies\label{sec:ch}}
Road networks have strong hierarchies since some roads are more important than others. For example, a highway allows high average speeds and therefore is a preferred connection between points in a road network in comparison to smaller roads. Contraction Hierarchies \cite{geisberger:2012} are a speed-up technique which exploits these hierarchies.

Contraction Hierarchies (CH) use a two phase approach. In the preprocessing phase, the CH is constructed given a graph $G=(V,E,\len)$ and a node order which sorts the nodes by importance. For example, an important node of a road network might be a node in a highway interchange, an unimportant node might be the head of a dead end. We denote the CH as $\ch$. The node order can be computed by searching for unimportant nodes \cite{geisberger:2012} or for important nodes first \cite{abraham:2012,delling:2014}. The nodes then are sorted into multiple levels of increasing importance. Two nodes of the same level must not have an edge between them. We obtain the CH $\ch$ by iteratively contracting the least important node according to the node order by adding shortcut edges between its neighbors.

An outgoing edge of a node to another node of a higher level is called an \emph{upward} edge, the opposite is called a \emph{downward} edge. A path which consists of only upward edges is called an \emph{up-path}, a path which consists of only downward edges is called a \emph{down-path}. Finally, a path which consists of an up-path, followed by a down-path, is called an \emph{up-down-path}. The node on an up-down-path with the highest level, i.e. the node which separates up-path and down-path, is called the \emph{middle node} $m$ of the path.

For every shortest path between node $s,t \in V$, there exists an up-down $s$-$m$-$t$ path in $\ch$ of the exact same length \cite{geisberger:2012}. Therefore, we can restrict the search to finding this exact path. We run a bidirectional search from $s$ and $t$. The forward search from $s$ may only use upward edges and finds the subpath $s$-$m$, the backward search from $t$ may only use the inverted downward edges and finds the subpath $m$-$t$. The search is stopped if the minimum queue key of both queues of forward and backward search is greater or equal to the tentative minimum distance $\distance(s,t)$ \cite{geisberger:2012}.


\section{Core Contraction Hierarchies\label{sec:core_ch}}
The core contraction hierarchy presents a compromise between constructing a full CH $\ch$ and running a bidirectional search on $G$. The core CH $\corech$ is obtained by stopping the iterative contraction of nodes of increasing importance during the construction of the CH early. This leads to a set $C \subseteq V$ of so-called core nodes which are uncontracted. The set of nodes $V$ therefore is separated into a set of core nodes $C$ and a set of contracted nodes $\chv = V \setminus C$. The graph $\ch = (\chv,\che,\mathfunction{len})$ therefore is a valid contraction hierarchy. It contains only contracted nodes and (shortcut) edges between those nodes. The graph $G_{C} = (V \setminus \chv,E \setminus \che,\mathfunction{len})$ is called the core graph.

The core CH query again is a bidirectional search from $s$ and $t$. The query represents a normal CH query while settling nodes $v \in \chv$, i.e., it consists of a forward search from $s$ using only upward edges and a backward search from $t$ using only inverted downward edges. If a search reaches an uncontracted core node, it considers all outgoing edges. Thus, the core CH query can be characterized as a CH query in $\ch$ which transitions into a full bidirectional search if it reaches the core $G_C$. We can use the same stopping criterion as for the CH query since the stopping criterion for a CH is more conservative than the stopping criterion for a pure bidirectional search.


\section{CH-Potentials\label{sec:ch_pot}}
CH-Potentials \cite{strasser:2021} are an extension of CH to efficiently calculate distances $\distance(v,t)$ for many nodes $v \in V$ to a fixed node $t$. CH-Potentials are based on PHAST \cite{delling:2011} which itself is an extension of CH to efficiently run all-to-one queries.

PHAST runs in two steps. The first step is a backward one-to-all search from $t$ using only inverted downward edges. This is the same as running the backward search of a CH query from $t$ without any stopping criterion. We obtain an array $B$ where $B[v]$ is the length of the shortest down path from $v$ to $t$ or $B[v]=\infty$ if there is no such path. We then iteratively calculate the distance $\distance(v,t)$ of nodes of the same level, starting at the highest level. This is possible since nodes of the same level must not have edges between them. To obtain the distances of nodes $u$ of the next lower level, we find the minimum $\min(B[u], \min_v(\len(u,v) + \distance(v,t)))$ for all upward edges $(u,v)$ of the node $v$. The distances $\distance(v,t)$ were already computed in the previous iteration.

We can use PHAST as to compute potentials if we run its two steps beforehand as a preprocessing step and then look up the respective distances. This preprocessing would be slow since it computes the distances to all nodes. CH-Potentials mitigate this problem by computing the results of the second PHAST step lazily while the first step remains the same. We do not calculate the distances of all nodes beforehand. To compute the potential of a node $u$, we recursively compute the potential for all nodes $v$ which are connected to $u$ by upward edges $(u,v) \in \che$. We then can compute the distance to $t$ as in the PHAST algorithm by finding $\min(B[u], \min_v(\len(u,v) + \distance(v,t)))$. We save all the calculated distances $\distance(v,t)$ in order to not compute any distance value twice.

CH-Potentials can be optimized further as \cite{strasser:2021} describes in detail.



\chapter{Problem and Definitions}\label{chapter:problem_definitions}
The long-haul truck driver routing problem is an extension to the common shortest path problem (SPP) which is defined as follows. Let $G=(V,E,\omega)$ be a graph where $V$ is the set of nodes, $E$ is the set of edges $(u,v)$ with $u,v \in V$, and $\omega$ is the weight function $\omega: E \rightarrow \mathbb{R}_{\ge 0}$ which assigns each edge a nonnegative weight or length. Given a start node $s \in V$ and a target node $t \in V$, the SPP searches a shortest path $p$ from $s$ to $t$, i.e., a path $p = \langle s=v_0,v_1,...,t=v_k, \rangle$ with $(v_i,v_{i+1}) \in E$ and minimal $len(p) = \sum_{i=0}^{k-1} \omega((v_{i}, v_{i+1}))$.

We introduce a set $P \subseteq V$ of parking nodes and a set $R$ of driving time constraints $r_i$. Each driving time constraint is defined by a maximum allowed driving time $r_{i,d}$ and a pause time $r_{i,p}$. Thereby, the driving time constraints define a relation $r_i \le r_{i+1}$ with $r_i \le r_j \implies r_{i,d} \le r_{j,d} \land  r_{i,p} \le r_{j,p} \forall i,j$. In words, a greater or equal driving time restriction has a longer or equal driving and pause time and there must be no restriction $r_i$ with a longer driving time limit, but shorter pause time than another restriction $r_j$. Before exceeding a driving time of $r_{i,d}$, the driver must stop and pause for a time of at least $r_{i,p}$. Afterwards, the driver is allowed to drive for a maximum time of $r_{i,d}$ again without stopping. Stops can only take place at nodes $v \in P$.

A path $p$ from $s$ to $t$ now includes not only the sequence of visited nodes $p = \langle s=v_0,v_1,...,t=v_k, \rangle$, but also a pause time $\rho: p \rightarrow \{0,r_{i,d}\}$ for each node i. It is $\rho(v_i) = 0$ $\forall v_i \notin P$.

\begin{definition}[Driving Time]
	The driving time $\drivingtime(p)$ of a path  $p = \langle s=v_0,v_1,...,t=v_k, \rangle$ is defined analogously to the length of a path in the ordinary SPP. It is $\drivingtime(p) = \sum_{i=0}^{k-1} \omega((v_{i}, v_{i+1}))$.
\end{definition}

\begin{definition}[Travel Time]
	The travel time $\traveltime(p)$ of a path  $p = \langle s=v_0,v_1,...,t=v_k, \rangle$ is defined as the sum of the driving time of the path and the total accumulated pause time $\traveltime(p) = \drivingtime(p) + \sum_{i=0}^{k} \rho(v_i)$.
\end{definition}

\begin{definition}[Path Compliance]
	A valid path $p = \langle s=v_0,v_1,...,t=v_k, \rangle$ must comply with the driving time restrictions $R$. The path $p$ complies with $R$ if it complies with all $r_l \in R$.

	Let $v_i$ be the starting node $s$ or any node on the path with $\rho(v_i) \ge r_{l,p}$ and let $v_j$ be the target node $t$ or any node on the path with $\rho(v_j) \ge r_{l,p}$ and $i < j$. Then let $q$ be the subpath of $p$ from $v_i$ to $v_j$. A path complies with driving time restriction $r_l$ if $\drivingtime(q) < r_{d,l}$ for all possible subpaths $q$ or there is a node $v_m$ on $q$ with $\rho(v_m) \ge r_{l,p}$ and $i < m < j$.
\end{definition}

The long-haul truck driver routing problem now can be defined as follows.

\begin{namedproblem}
	\problemtitle{\textsc{Long-Haul Truck Driver Routing}}
	\probleminput{A graph $G=(V,E,\omega)$, a set of parking nodes $P \subseteq V$, a set of driving time constraints $R$, and start and target nodes $s,t \in V$}
	\problemquestion{Find the path $p$ from $s$ to $t$ in $G$ which minimizes travel time $\traveltime(p)$ and complies with the driving time restrictions $R$.}
\end{namedproblem}

In many practical applications, the number of different driving time constraints is limited to only one or two constraints, i.e., $|R| = 1$ or  $|R| = 2$. Therefore, we will often only consider one of these special cases.
%% content.tex
%%

%% ==============
\chapter{Algorithm}
\label{ch:Algorithm}
%% ==============
In this chapter, we introduce a labeling algorithm which solves the long-haul truck driver routing problem. At first, we will restrict the problem to one driving time constraint for simplicity and drop that constraint later. We then describe extensions of the base algorithm to achieve better runtimes on road networks.

\section{Dijkstra's Algorithm with One Driving Time Constraint\label{sec:dijkstra_csp}}
Dijkstra's algorithm solves the shortest path problem by maintaining a queue $Q$ of nodes with ascending tentative distance from the starting node $s$, and iteratively settling the node with the smallest distance. When Dijkstra settles a node $u$, it tests if the distance to the neighbor nodes $v$ with $(u,v) \in E$ can be improved by choosing the current node as a predecessor. We say it \emph{relaxes} all edges $(u,v) \in E$. The search can be stopped if the target node $t$ was removed from the queue \cite{dijkstra:1959}.

We will adapt Dijkstra's algorithm for solving the long-haul truck driver routing problem with one driving time constraint $\restrset = \{\restr\}$ and abbreviate this restriction of the problem \emph{1-DTC}. While Dijkstra's algorithm manages a queue of nodes and assigns each node one tentative distance, our algorithm manages a queue $Q$ of labels and a set $L(v)$ of labels for each node $v \in V$.

Labels represent a possible route, respectively a possible solution for a query from $s$ to the node they belong to. A label in a label set $L(v)$ may represent suboptimal routes to $v$, i.e., routes which are not a shortest route between $s$ and $v$. A label set never contains labels which represent routes with an invalid path according to $\restr$. A label $l$ contains

\begin{itemize}
	\item $\concretett(l)$, the total travel time from the starting node $s$
	\item $\breakDist(l)$, the driving time since the last break
	\item $\pred(l)$, its preceding label
\end{itemize}

\subsection{Settling a Label}
In contrast to Dijkstra, the search \emph{settles} a label $l \in L(u)$ in each iteration instead of a node $u$. When settling a label, the search first removes $l$ from the queue. Similar to Dijkstra, it then relaxes all edges $(u,v) \in E$ with $l \in L(u)$ as shown in fig. \ref{alg:settle_next_label}.

\begin{figure}[hbtp]
	\setlength{\interspacetitleruled}{0pt}%
	\setlength{\algotitleheightrule}{0pt}%
	\begin{algorithm*}[H]
		\SetFuncSty{textsc}
		\DontPrintSemicolon
		\SetKwData{Q}{Q}
		\SetKwData{L}{L}

		\SetKwFunction{settleNextLabel}{settleNextLabel}
		\SetKwFunction{relaxEdge}{relaxEdge}
		\SetKwFunction{queueDeleteMin}{deleteMin}

		\SetKwProg{Pn}{Procedure}{:}{\KwRet}
		\Pn{\settleNextLabel{}}{
			$l \leftarrow$ \Q.\queueDeleteMin{} \;
			\BlankLine

			\ForAll{ $(u,v) \in E$ }
			{
				\relaxEdge{$(u,v),l$}
			}
		}
	\end{algorithm*}
	\setlength{\interspacetitleruled}{2pt}%
	\setlength{\algotitleheightrule}{\algotitleheightruledefault}%

	\caption{\label{alg:settle_next_label}Settling a label $l \in L(u)$ removes the label from the queue and relaxes all the outgoing edges of $u$.}
\end{figure}

Relaxing an edge consists of the three steps label \emph{propagation}, \emph{pruning} and \emph{dominance} checks.

\paragraph{Label Propagation.}
Labels can be propagated along edges. Let $l \in L(u)$ be a label at $u$ and $(u,v) = e \in E$, then $l$ can be propagated to $v$ resulting in a label $l'$ with $\concretett(l') = \concretett(l) + \mathfunction{len}(e)$, $\breakDist(l') = \breakDist(l) + \mathfunction{len}(e)$, and $\pred(l') = l$.

\paragraph{Label Pruning.}
After propagating a label, we discard the label if it violates the driving time constraint $\restr$. That is, if $\breakDist(l) > \restr_d$.


\paragraph{Label Dominance}
In general, it is no longer clear when a label presents a better solution than another label since it now contains two distance values. A label $l$ at a node $v$ might represent a shorter route from $s$ to $v$ than another label $m$ but might have shorter remaining driving time budget $\restr_d - \breakDist(l)$. The label $l$ yields a better solution for a query $s$-$v$, but this does not imply that it is part of a better solution for a query from $s$-$t$. It might not even yield a valid path to $t$ at all while $m$ reaches the target due to the greater remaining driving time budget. In one case we can proof that a label $l \in L(v)$ cannot yield a better solution than a label $m \in L(v)$. We say $m$ \emph{dominates} $l$.

\begin{definition}[Label Dominance for 1-DTC]
	A label $l \in L(v)$ dominates another label $m \in L(v)$ if $\concretett(m) \ge \concretett(l)$ and $\breakDist(m) \ge \breakDist(l)$.
\end{definition}

If a label $l \in L(v)$ is dominated by another label $m \in L(v)$, then $m$ represents a route from $s$ to $t$ with a shorter or equal total travel time and longer or equal remaining driving time budget until the next break. Therefore, in each solution which uses the label $l$, $l$ can trivially be replaced by the label $m$. The solution will still comply with the driving time constraint $\restr$ and yield a shorter or equal total travel time, so we are allowed to simply discard dominated labels in our search.

\begin{definition}[Pareto-Optimal Label]
	A label $l \in L(v)$ is pareto-optimal if it is not dominated by any other label $m \in L(v)$.
\end{definition}

A label $l$ will only be inserted into a label set $L(v)$ if it is pareto-optimal. If a label $l$ is inserted into $L(v)$, labels $m \in L(v)$ are removed from $L(v)$ if $l$ dominates them. $L(v)$ therefore is the set of known pareto-optimal solutions at $v$. In fig. \ref{alg:remove_dominated} we define the procedure \textsc{removeDominated($l$)} as an operation on a label set.

\begin{figure}[hbtp]
	\setlength{\interspacetitleruled}{0pt}%
	\setlength{\algotitleheightrule}{0pt}%
	\begin{algorithm*}[H]
		\SetFuncSty{textsc}
		\SetKwFor{ForAll}{forall}{do}
		\DontPrintSemicolon
		\SetKwData{L}{L}

		\SetKwFunction{removeDominated}{removeDominated}
		\SetKwFunction{queueRemove}{remove}

		\SetKwProg{Pn}{Procedure}{:}{\KwRet}
		\Pn{\removeDominated{$l$}}{
			\ForAll{$m \in L$}
			{
				\If{\text{$l$ dominates $m$}}{
					\L.\queueRemove($m$)\;
				}
			}
		}
	\end{algorithm*}
	\setlength{\interspacetitleruled}{2pt}%
	\setlength{\algotitleheightrule}{\algotitleheightruledefault}%

	\caption{\label{alg:remove_dominated}The procedure $L$.\textsc{removeDominated}($l$) removes all labels from the label set $L$ which are dominated by the label $l$.}
\end{figure}

We now use \textsc{removeDominated} to define the procedure \textsc{relaxEdge$'$} as described in fig. \ref{alg:relax_edge_no_p} which propagates a label along an edge and updates the neighbor node's label set if necessary.

\begin{figure}[hbtp]
	\setlength{\interspacetitleruled}{0pt}%
	\setlength{\algotitleheightrule}{0pt}%
	\begin{algorithm*}[H]
		\SetFuncSty{textsc}
		\DontPrintSemicolon
		\SetKwData{Q}{Q}
		\SetKwData{L}{L}

		\SetKwFunction{relaxEdgeNoP}{relaxEdge$'$}
		\SetKwFunction{queueDeleteMin}{deleteMin}

		\SetKwData{Q}{Q}
		\SetKwData{dist}{d}
		\SetKwData{L}{L}
		\SetKwData{pred}{pred}
		\SetKwArray{ds}{ds}
		\SetKwFunction{queueDeleteMin}{deleteMin}
		\SetKwFunction{queueInsert}{queueInsert}
		\SetKwFunction{setInsert}{insert}
		\SetKwFunction{queueMin}{min}
		\SetKwFunction{queueMinKey}{minKey}
		\SetKwFunction{queueDecreaseKey}{decreaseKey}
		\SetKwFunction{queueContains}{contains}
		\SetKwFunction{listInsert}{insert}
		\SetKwFunction{removeDom}{removeDominated}

		\SetKwProg{Pn}{Procedure}{:}{\KwRet}
		\Pn{\relaxEdgeNoP{(u,v), l}}{
			\If{$\breakDist(l) + \mathfunction{len}(u,v) \le \restr_d$}
			{
				$l' \leftarrow \{(\concretett(l) + \mathfunction{len}(u,v), \breakDist(l) + \mathfunction{len}(u,v)), l\}$\;
				\If{$l'$ is not dominated by any label in \L{$v$}}
				{
					\L{$v$}.\removeDom{$l'$} \;
					\L{$v$}.\setInsert{$l'$} \;
					\Q.\queueInsert{$\concretett(l')$, $l'$}
				}
			}
		}
	\end{algorithm*}
	\setlength{\interspacetitleruled}{2pt}%
	\setlength{\algotitleheightrule}{\algotitleheightruledefault}%

	\caption{\label{alg:relax_edge_no_p}Relaxing an edge $(u,v) \in E$ when settling a label $l \in L(u)$.}
\end{figure}

\subsection{Parking at a Node}
The procedure \textsc{relaxEdge$'$} does not account for parking nodes. When propagating a label $l \in L(u)$ along an edge $(u,v) \in E$ and $v \in P$, we have to consider pausing at $v$. Since we do not know if pausing at $v$ or continuing without a break is the better solution, we generate both labels and add them to label set $L(v)$ and the queue $Q$ as defined in fig. \ref{alg:relax_edge}.

\begin{figure}[hbtp]
	\setlength{\interspacetitleruled}{0pt}%
	\setlength{\algotitleheightrule}{0pt}%
	\begin{algorithm*}[H]
		\SetFuncSty{textsc}
		\DontPrintSemicolon
		\SetKwData{Q}{Q}
		\SetKwData{L}{L}

		\SetKwFunction{relaxEdge}{relaxEdge}
		\SetKwFunction{queueDeleteMin}{deleteMin}

		\SetKwData{Q}{Q}
		\SetKwData{L}{L}
		\SetKwData{D}{D}
		\SetKwData{pred}{pred}
		\SetKwArray{ds}{ds}
		\SetKwFunction{queueDeleteMin}{deleteMin}
		\SetKwFunction{queueInsert}{queueInsert}
		\SetKwFunction{setInsert}{insert}
		\SetKwFunction{queueMin}{min}
		\SetKwFunction{queueMinKey}{minKey}
		\SetKwFunction{queueDecreaseKey}{decreaseKey}
		\SetKwFunction{queueContains}{contains}
		\SetKwFunction{listInsert}{insert}
		\SetKwFunction{removeDom}{removeDominated}

		\SetKwProg{Pn}{Procedure}{:}{\KwRet}
		\Pn{\relaxEdge{(u,v), l}}{
			$\D \leftarrow \{\}$\;
			\If{$\breakDist(l) + \mathfunction{len}(u,v) \le \restr_d$}
			{
				\D.\setInsert{$(\concretett(l) + \mathfunction{len}(u,v), \breakDist(l) + \mathfunction{len}(u,v), l)$}\;

				\If{$v \in P$}
				{
					\D.\setInsert{$(\concretett(l) + \mathfunction{len}(u,v) + \restr_b, 0, l)$}\;
				}

				\ForAll{ $l' \in D$ }
				{
					\If{$l'$ is not dominated by any label in \L{$v$}}
					{
						\L{$v$}.\removeDom{$l'$} \;
						\L{$v$}.\setInsert{$l'$} \;
						\Q.\queueInsert{$\concretett(l')$,$l'$}
					}
				}
			}
		}
	\end{algorithm*}
	\setlength{\interspacetitleruled}{2pt}%
	\setlength{\algotitleheightrule}{\algotitleheightruledefault}%

	\caption{\label{alg:relax_edge}Relaxing an edge $(u,v) \in E$ when settling a label $l \in L(u)$ with regard to parking nodes.}
\end{figure}

\subsection{Initialization and Stopping Criterion}
We initialize the label set $L(s)$ of $s$ and the queue $Q$ with a label which only contains distances of zero and a dummy element as a predecessor. We stop the search when $t$ was removed from $Q$. The definition of the final algorithm \ref{alg:CSP} \textsc{Dijkstra+1-DTC} is now trivial.

\begin{algorithm}[bt]
	\caption{\textsc{Dijkstra+1-DTC}}\label{alg:CSP}

	% Some settings
	\DontPrintSemicolon %dontprintsemicolon
	\SetFuncSty{textsc}
	\SetKwFor{ForAll}{forall}{do}

	% Declaration of data containers and functions
	\SetKwData{Q}{Q}
	\SetKwData{dist}{d}
	\SetKwData{L}{L}
	\SetKwData{pred}{pred}
	\SetKwArray{ds}{ds}
	\SetKwFunction{queueDeleteMin}{deleteMin}
	\SetKwFunction{queueInsert}{queueInsert}
	\SetKwFunction{setInsert}{insert}
	\SetKwFunction{queueMin}{min}
	\SetKwFunction{queueMinKey}{minKey}
	\SetKwFunction{queueDecreaseKey}{decreaseKey}
	\SetKwFunction{queueContains}{contains}
	\SetKwFunction{listInsert}{insert}
	\SetKwFunction{removeDom}{removeDominated}
	\SetKwFunction{settleNextLabel}{settleNextLabel}

	% Algorithm interface
	\KwIn{Graph $G=(V,E,\mathfunction{len})$, set of parking nodes $P \subseteq V$, set of driving time constraints $\restrset=\{r\}$, start and target nodes $s,t \in V$}
	\KwData{Priority queue \Q, per node set \L{$v$} of labels for all $v \in V$}
	\KwOut{Shortest route with $\concretett(j) = \traveltime(s,t)$}

	% The algorithm
	\BlankLine
	\tcp{Initialization}
	\Q.\queueInsert{$0$,$(0,0,\bot)$}\;
	\L{$s$}.\setInsert{$(0,0,\bot)$}\;
	\BlankLine
	\tcp{Main loop}
	\While{\Q is not empty}
	{
		\settleNextLabel{}\;

		\If{\text{label at $t$ was settled}}
		{
			\Return\;
		}
	}
\end{algorithm}

\subsection{Correctness\label{sec:dijkstra_csp_correctness}}
An $s$-$t$ query with the original Dijkstra's algorithm can be stopped when $t$ was removed from the queue since all of the following nodes in the queue have larger distances and edge lengths are non-negative by definition. Therefore, relaxing an outgoing edge of these nodes cannot lead to an improvement of the distance at $t$.
In our case, the labels in the queue are ordered by their travel time. Relaxing an edge can only increase the travel time since both, edge lengths and break times are non-negative. Therefore, the same argument as for the original Dijkstra's algorithm applies and the first label at $t$ which was removed from the queue contains the shortest travel time $\traveltime(s,t)$.

\section{Goal Directed Search with One Driving Time Constraint}
In this section, we transform the base algorithm described in section \ref{sec:dijkstra_csp} to a goal-directed search with the A* algorithm. We introduce a new potential $\concretepotential_t$ which is based on the CH potential $\chpotential_t$ which is described in section \ref{sec:ch_pot}. We then show that we still can stop the search when the first label at $t$ is removed from the queue.

The difference between Dijkstra and A* is the order in which nodes are being removed from the queue. In our case, this corresponds to the order of labels being removed from the queue. Instead of using their travel time $\concretett(l)$ as a queue key, a label $l \in L(v)$ is added to the queue with the key $\concretett(l) + \concretepotential_t(l,v)$. As shown in algorithm \ref{alg:CSPPot}, the adaption of the pseudocode of the coarse algorithm is trivial.

\begin{algorithm}[bt]
	\caption{\textsc{A*+1-DTC}}\label{alg:CSPPot}

	% Some settings
	\DontPrintSemicolon %dontprintsemicolon
	\SetFuncSty{textsc}
	\SetKwFor{ForAll}{forall}{do}

	% Declaration of data containers and functions
	\SetKwData{Q}{Q}
	\SetKwData{dist}{d}
	\SetKwData{L}{L}
	\SetKwData{pred}{pred}
	\SetKwArray{ds}{ds}
	\SetKwFunction{queueDeleteMin}{deleteMin}
	\SetKwFunction{queueInsert}{queueInsert}
	\SetKwFunction{setInsert}{insert}
	\SetKwFunction{queueMin}{min}
	\SetKwFunction{queueMinKey}{minKey}
	\SetKwFunction{queueDecreaseKey}{decreaseKey}
	\SetKwFunction{queueContains}{contains}
	\SetKwFunction{listInsert}{insert}
	\SetKwFunction{removeDom}{removeDominated}
	\SetKwFunction{settleNextNode}{settleNextNode}

	% Algorithm interface
	\KwIn{Graph $G=(V,E,\mathfunction{len})$, set of parking nodes $P \subseteq V$, a set of driving time constraints $\restrset = \{r\}$, start and target nodes $s,t \in V$, potential $\concretepotential_t()$}
	\KwData{Priority queue \Q, per node set \L{$v$} of labels for all $v \in V$}
	\KwOut{Shortest route with $\concretett(j) = \traveltime(s,t)$}

	% The algorithm
	\BlankLine
	\tcp{Initialization}
	$l_s \leftarrow (0,0,\bot)$\;
	\Q.\queueInsert{$\concretepotential_t((l_s),s)$, $l_s$}\;
	\L{$s$}.\setInsert{$l_s$}\;
	\BlankLine
	\tcp{Main loop}
	\While{\Q is not empty}
	{
		\settleNextNode{}\;

		\If{\text{minimum of $Q$ is label at $t$}}
		{
			\Return\;
		}
	}
\end{algorithm}

The only thing left is the adaption of \textsc{relaxEdge} in fig. \ref{alg:relax_edge} where we change the queue keys to use $\concretett(l) + \concretett(l,v)$ instead. The result is shown in fig. \ref{alg:relax_edge_a_star}.

\begin{figure}[hbtp]
	\setlength{\interspacetitleruled}{0pt}%
	\setlength{\algotitleheightrule}{0pt}%
	\begin{algorithm*}[H]
		\SetFuncSty{textsc}
		\DontPrintSemicolon
		\SetKwData{Q}{Q}
		\SetKwData{L}{L}

		\SetKwFunction{relaxEdge}{relaxEdge}
		\SetKwFunction{queueDeleteMin}{deleteMin}

		\SetKwData{Q}{Q}
		\SetKwData{L}{L}
		\SetKwData{D}{D}
		\SetKwData{pred}{pred}
		\SetKwArray{ds}{ds}
		\SetKwFunction{queueDeleteMin}{deleteMin}
		\SetKwFunction{queueInsert}{queueInsert}
		\SetKwFunction{setInsert}{insert}
		\SetKwFunction{queueMin}{min}
		\SetKwFunction{queueMinKey}{minKey}
		\SetKwFunction{queueDecreaseKey}{decreaseKey}
		\SetKwFunction{queueContains}{contains}
		\SetKwFunction{listInsert}{insert}
		\SetKwFunction{removeDom}{removeDominated}

		\SetKwProg{Pn}{Procedure}{:}{\KwRet}
		\Pn{\relaxEdge{(u,v), l}}{
			\If{$\breakDist(l) + \mathfunction{len}(u,v) < \restr_d$}
			{
				\D.\setInsert{$(\concretett(l) + \mathfunction{len}(u,v), \breakDist(l) + \mathfunction{len}(u,v), l)$}\;

				\If{$v \in P$}
				{
					\D.\setInsert{$(\concretett(l) + \mathfunction{len}(u,v) + \breakDist_p, 0, l)$}\;
				}

				\ForAll{ $l' \in D$ }
				{
					\If{$l'$ is not dominated by any label in \L{$v$}}
					{
						\L{$v$}.\removeDom{$l'$} \;
						\L{$v$}.\setInsert{$l'$} \;
						\Q.\queueInsert{$\concretett(l') + \concretepotential_t(l')$, l'}
					}
				}
			}
		}
	\end{algorithm*}
	\setlength{\interspacetitleruled}{2pt}%
	\setlength{\algotitleheightrule}{\algotitleheightruledefault}%

	\caption{\label{alg:relax_edge_a_star} Relaxing an edge with regard to the potential.}
\end{figure}

\subsection{Potential for One Driving Time Constraint}\label{section:potential_csp}
Given a target node $t$, the CH potential $\chpotential_t(v)$ yields a perfect estimate for the distance $\distance(v,t)$ from $v$ to $t$ without regard for driving time constraints and breaks. This trivially is a lower bound for the remaining travel time for any label at $v$. A better lower bound for the remaining travel time of a label at $v$ to $t$, including breaks due to the driving time limit, can be calculated by taking the minimum necessary amount of breaks into account. We define \minBreaks($d$) as a function which calculates the minimum amount of necessary breaks given a driving time $d$.

\begin{align}\label{eq:min_breaks}
	\minBreaks(d) = \begin{dcases}
		\ceil*{ \frac{d}{\restr_d} } - 1 & d > 0 \\
		0                                & else
	\end{dcases}
\end{align}

Simply using $\floor*{ \frac{d}{\restr_d} }$ is not sufficient since we do not need to pause for a driving time of exactly $\restr_d$. We now can calculate a lower bound for the minimum necessary break time given a driving time $d$

\begin{align}\label{eq:min_break_time}
	\minBreakTime(d) = \minBreaks(d) \cdot \restr_b
\end{align}

and finally define our node potential as

\begin{align}
	\concretepotential{'}_t(v) & = \minBreakTime(d) + \chpotential_t(v) \\
	                           & = \minBreakTime(d) + \distance(v,t)
\end{align}

A node potential is called \emph{feasible} if it does not overestimate the distance of any edge in the graph, i.e.

\begin{align}
	\label{eq:node_potential_feasibility}
	len(u,v) - \concretepotential_t(u) + \concretepotential_t(v) \ge 0 \quad \forall (u,v) \in E
\end{align}

A feasible node potential allows us to stop the A* search when the node $t$, respectively the first label at $t$, was removed from the queue. Following counterexample of a query using the graph in Fig. \ref{fig:graph_infeasible_potential} shows that $\concretepotential{'}_t$ is not feasible. With a driving time limit of 6 and a break time of 1, the potential here will yield a value $\concretepotential_t(s) = 8$ since the potential includes the minimum required break time for a path from s to t. Consequently, with $\concretepotential{'}_t(v) = 5$ and $len(s,v) = 2$, $len(s,v) - \concretepotential{'}_t(s) + \concretepotential{'}_t(v) = -1$.

\begin{figure}[hbtp]
	\centering
	\tikzstyle{node}=[circle,inner sep=0.5mm,minimum size=5.25mm,draw = black]
\tikzstyle{bright}=[fill=black!14]
\tikzstyle{dark}=[fill=black!28]
\tikzstyle{lightEdgeStyle}=[black!20]

\begin{tikzpicture}[scale=1.5, bend angle = 20]

	% Obere Reihe
	\node(s) at (1,1) [node, bright] {s};
	\node(v) at (2,1) [node, bright] {v};
	\node(t) at (3,1) [node, bright] {t};

	\draw[->] (s) -> node[midway, above]{2} (v);
	\draw[->] (v) -> node[midway, above]{5} (t);
	% \foreach \i [evaluate = \i as \lastNode using \i-1] in {2,3,...,\numberOfNodes}
	% 	{
	% 		\node (Top\i) at (\i,1) [node, bright] {\i}
	% 		edge[<-] (Top\lastNode);
	% 	}


	% % Pfeile nach rechts
	% \pgfmathparse{\numberOfNodes - 2}
	% \foreach \i [evaluate = \i as \nextNode using \i+2] in {1,2,...,\pgfmathresult}
	% 	{
	% 		\foreach \j [count=\nodeIndex from \nextNode] in {\nextNode,...,\numberOfNodes}
	% 			{
	% 				\draw[->, lightEdgeStyle] (Bot\i) to [bend right] (Bot\nodeIndex);
	% 			}
	% 	}

\end{tikzpicture}

	\caption{A graph with the potential to break the potential.}
	\label{fig:graph_infeasible_potential}
\end{figure}

A variant of the potential accounts for the driving time since the last break of a label $\breakDist(l)$ to calculate the minimum required break time on the $v$-$t$ path.

\begin{align}
	\begin{split}
		\concretepotential_t(l,v) & = \minBreakTime(\breakDist(l) + \chpotential(v)) +\chpotential(v) \\
		& = \minBreakTime(\breakDist(l) + \distance(v,t)) + \distance(v,t)
	\end{split}
\end{align}

Since the potential now uses information from a label $l$ with $l \in L(v)$, it no longer is a node potential but also depends on the chosen label at $v$. The feasibility definition as defined in inequality \ref{eq:node_potential_feasibility} can no longer be applied. We therefore have to show that queue keys of labels, which represent a lower bound estimate for the travel time of the entire route, can only increase over time.

\begin{lemma}\label{lemma:pot_labels_get_larger}
	Let $p = \langle s=v_0,v_1,\ldots,t=v_k \rangle$ be a path with labels $l_i$ at nodes $v_i$. Then $\concretett(l_{i-1}) + \concretepotential_t(l_{i-1},v_{i-1}) \le \concretett(l_i) + \concretepotential_t(l_i,v_i)$.
\end{lemma}

\begin{proof}
	Let $(u,v) \in E$ be an edge. The procedure \textsc{relaxEdge} in fig. \ref{alg:relax_edge_a_star} can produce two new labels at a node $v$ for each label at $u$, depending on if $v$ is a parking node. We differentiate the two cases not parking at $v$ and parking at $v$. Let $l \in L(u)$ and $l' \in L(v)$.

	Following general observations can be made:

	\begin{enumerate}
		\item $d \ge d' \implies \minBreakTime(d) \ge \minBreakTime(d')$
		\item $\minBreakTime(d + \restr_d) = \restr_b + \minBreakTime(d)$
		\item $\restr_d \ge \breakDist(l') \ge \breakDist(l) + \len(u,v) \ge \breakDist(l)$\\(line $2$ in \textsc{relaxEdge} in fig. \ref{alg:relax_edge_a_star})
		\item   $\len(u,v) - \chpotential_t(u)+ \chpotential_t(v) \ge 0$ (feasibility of the CH potential)
		\item  $\len(u,v) + \chpotential_t(v) \ge \chpotential_t(u)$
	\end{enumerate}

	We show that $\concretett(l') + \concretepotential_t(l',v) - \concretett(l) - \concretepotential_t(l,u) \ge 0$.

	\emph{Case 1: Not parking at $v$.} In this case, $\concretett(l') = \concretett(l) + \len(u,v)$ and $\breakDist(l') = \breakDist(l) + \len(u,v)$.

	\begin{align}
		\begin{split}\label{eq:label_feasibility_proof_1}
			&\concretett(l') - \concretett(l) - \concretepotential_t(l,u) + \concretepotential_t(l',v)\\
			&= \concretett(l) + \len(u,v) - \concretett(l)\\
			& \phantom{{}=1} - (\minBreakTime(\breakDist(l)+\chpotential(u)) + \chpotential(u))\\
			& \phantom{{}=1} + \minBreakTime(\breakDist(l')+\chpotential(v)) + \chpotential(v)\\
			&= \len(u,v) + \minBreakTime(\breakDist(l')+\chpotential(v))\\
			& \phantom{{}=1} - \minBreakTime(\breakDist(l)+\chpotential(u)) - \chpotential(u) + \chpotential(v)\\
			&= \len(u,v) + \minBreakTime(\breakDist(l) + \len(u,v) + \chpotential(v))\\
			& \phantom{{}=1} - \minBreakTime(\breakDist(l)+\chpotential(u)) - \chpotential(u) + \chpotential(v) \\
			&\stackrel{\text{(1. and 5.)}}{\ge} \len(u,v) + \minBreakTime(\breakDist(l) + \chpotential(u))\\
			& \phantom{{}=1} - \minBreakTime(\breakDist(l)+\chpotential(u)) - \chpotential(u) + \chpotential(v) \\
			&= \len(u,v) - \chpotential(u) + \chpotential(v)\\
			&\stackrel{\text{(4.)}}{\ge} 0
		\end{split}
	\end{align}

	\emph{Case 2: Parking at $v$.} In this case, $\concretett(l') = \concretett(l) + \len(u,v) + \restr_b$ and $\breakDist(l') = 0$.

	\begin{align}
		\begin{split}\label{eq:label_feasibility_proof_2}
			&\concretett(l') - \concretett(l) - \concretepotential_t(l,u) + \concretepotential_t(l',v)\\
			&= \concretett(l) + \len(u,v) + \restr_b - \concretett(l)\\
			& \phantom{{}=1} - (\minBreakTime(\breakDist(l) + \chpotential(u)) + \chpotential(u))\\
			& \phantom{{}=1} + \minBreakTime(\chpotential(v)) + \chpotential(v)\\
			&=  \len(u,v) + \restr_b + \minBreakTime(\chpotential(v))\\
			& \phantom{{}=1} - \minBreakTime(\breakDist(l)+\chpotential(u)) - \chpotential(u) + \chpotential(v)\\
			&\stackrel{\text{(2.)}}{=}  \len(u,v) + \minBreakTime(\restr_d + \chpotential(v))\\
			& \phantom{{}=1} - \minBreakTime(\breakDist(l)+\chpotential(u)) - \chpotential(u) + \chpotential(v)\\
			&\stackrel{\text{(1. and 3.)}}{\ge} \len(u,v) + \minBreakTime(\breakDist(l) + \len(u,v) + \chpotential(v))\\
			& \phantom{{}=1} - \minBreakTime(\breakDist(l)+\chpotential(u)) - \chpotential(u) + \chpotential(v)\\
			&\stackrel{\text{(1. and 4.)}}{\ge} \len(u,v) + \minBreakTime(\breakDist(l) + \chpotential(u))\\
			& \phantom{{}=1} - \minBreakTime(\breakDist(l)+\chpotential(u)) - \chpotential(u) + \chpotential(v)\\
			&=  \len(u,v) - \chpotential(u) + \chpotential(v)\\
			&\stackrel{\text{(4.)}}{\ge} 0
		\end{split}
	\end{align}
\end{proof}

\begin{lemma}\label{lemma:pot_lower_bound_csp}
	The sum $\concretett(l) + \concretepotential_t(l,v)$ of a label $l$ at a node $v$ is a lower bound for the travel time from $s$ to $t$ using $l$.
\end{lemma}

\begin{proof}
	Let $p = \langle s=v_0,v_1,\ldots,t=v_k \rangle$ be a path with labels $l_i$ at nodes $v_i$. With lemma \ref{lemma:pot_labels_get_larger} and $\concretepotential_t(l_k,t) = 0$ follows

	\begin{align*}
		\concretett(l_{i}) + \concretepotential_t(l_{i},v_{i}) & \le \concretett(l_{i+1}) + \concretepotential_t(l_{i+1},v_{i+1})     \\
		                                                       & \le \dots \le \concretett(l_{k}) + \concretepotential_t(l_{k},v_{k}) \\
		                                                       & = \concretett(p)
	\end{align*}

\end{proof}

\begin{theorem}\label{theorem:pot_stop_criterion}
	The search can be stopped when the first label at $t$ is removed from the queue.
\end{theorem}

\begin{proof}
	When a label $l_t$ at $t$ is removed from the queue during an $s$-$t$ query, all remaining labels $l$ at nodes $v$ in the queue fulfill $\concretett(t) + \concretepotential_t(l_t,t) \le \concretett(v) + \concretepotential_t(l,v)$. The same holds for all labels which will be inserted into the queue at a later point in time (lemma \ref{lemma:pot_labels_get_larger}). Assume that $\concretett(l_t)$ is not the shortest route from $s$ to $t$ with time $\traveltime(s,t)$. Then, a shorter route exists which uses at least one unsettled label $l \in L(v)$ at a node $v$. With lemma \ref{lemma:pot_lower_bound_csp} and $\concretepotential_t(l_t,t) = 0$ follows $\concretett(t) = \concretett(t) + \concretepotential_t(l_t,t) \le \concretett(v) + \concretepotential_t(l,v) \le \concretett(p)$ which contradicts the assumption that $p$ yields a shorter $s$-$t$ travel time. Therefore, it must be $\concretett(l_t) = \traveltime(s,t)$ when $l_t$ was removed from the queue and the search can be stopped.
\end{proof}

\section{Multiple Driving Time Constraints}
Dijkstra's algorithm with one driving time constraint (1-DTC) can easily be adapted to handle multiple driving time constraints $\restr_i$. We abbreviate the generalized problem \emph{n-DTC}. For a number of $n$ driving time constraints, a label now contains the total travel time $\concretett$ and $n$ driving time values $\breakDist_1, \ldots , \breakDist_n$. Each $\breakDist_i$ represents the driving time since the last break at a node $v$ with break time $\breakTime(v) \ge \restr_{i,b}$. Pausing at a node occurs with one of the available break times $\restr_{i,b}$ of a driving time constraint $\restr_i \in \restrset$. Pausing with an arbitrary break time is possible but yields longer travel times and no advantage. When a path breaks at $v$ for a time $\restr_{i,b}$, the corresponding label $l \in L(v)$ has $\breakDist(l) = 0$ for all $0 < j \le i$ since the breaks with shorter break times are \emph{included} in the longer break.

\paragraph{Label Propagation}
Label propagation simply extends the component-wise addition of the edge weight to all elements of the distance vector. Let $l \in L(u)$ be a label at $u$ and $(u,v) = e \in E$, then $l$ can be propagated to $v$ resulting in a label $l'$ with $\breakDist_i(l') = \breakDist_i(l) + \mathfunction{len}(e)$ $\forall i \le |\restrset|$, and $\pred(l') = l$.

\paragraph{Label Pruning}
The pruning rule for driving time constraints is generalized in a similar way. A label is discarded if $\breakDist_i(l) > \restr_{i,d}$ for any $i$ with $0 < i \le |\restrset|$.


\paragraph{Label Dominance}
At last, label dominance can be generalized to multiple driving time constraints as follows.
\begin{definition}[Label Dominance]
	A label $l \in L(v)$ dominates another label $m \in L(v)$ if $\breakDist_i(m) \ge \breakDist_i(l)$ $\forall i \le |\restrset|$.
\end{definition}

\subsection{Potential for Multiple Driving Time Constraints\label{section:potential_n_csp}}
TODO: Rewrite for only two dtc?
In section \ref{section:potential_csp} we defined the potential $\concretepotential_t(l,v)$ to extend Dijkstra to an A* search with one driving time constraint. We will now generalize $\concretepotential_t$ for the use with an arbitrary number $n$ of driving time constraints.

In eq. \refeq{eq:min_breaks} we used the distance $\distance(v,t)$ without regard for pausing from $v$ to $t$ and the driving time $\breakDist(l)$ since the last break on the route to calculate a lower bound for the amount of necessary breaks until we reach the target node. We now have to calculate the lower bound with respect to all driving time constraints. How many breaks of which duration do we need at least to comply with all driving time constraints $\restr_i$? For longer driving time constraints, we will always need a greater or equal amount of breaks than for shorter driving time constraints since they have a longer maximum allowed driving time $\restr_{i,d}$. At the same time, a break of length $\restr_{i,b}$ will also include breaks of lengths $\restr_{j,b}$ with $j < i$. We start with simply calculating the amount of necessary breaks $\minBreaks_i(d)$ for all restrictions $r_i$.

\begin{align}\label{eq:min_breaks_n}
	\minBreaks_i(d) = \begin{dcases}
		\ceil*{ \frac{d}{\restr_{i,d}} } - 1 & d > 0 \\
		0                                    & else
	\end{dcases}
\end{align}

Consider the example graph in fig. \ref{fig:graph_short_long_break} with two driving time constraints with permitted driving times of $4$ and $9$. Since the distance $\distance(s,t)$ is $10$, a route must have at least one long and two short breaks. If the long break is made at $u$, only one additional short break must be made at $w$. The long break made one shorter break obsolete. To obtain a lower bound for the amount of breaks for a restriction $r_i$, we therefore must subtract the amount of longer breaks.

\begin{figure}[hbtp]
	\centering
	\tikzstyle{node}=[circle,inner sep=0.5mm,minimum size=5.25mm,draw = black]
\tikzstyle{bright}=[fill=black!14]
\tikzstyle{dark}=[fill=black!28]
\tikzstyle{lightEdgeStyle}=[black!20]

\begin{tikzpicture}[scale=1.5, bend angle = 20]

	% Obere Reihe
	\node(s) at (1,1) [node, bright] {s};
	\node(u) at (2,1) [node, bright] {u};
	\node(v) at (3,1) [node, bright] {v};
	\node(w) at (4,1) [node, bright] {w};
	\node(t) at (5,1) [node, bright] {t};

	\draw[->] (s) -> node[midway, above]{4} (u);
	\draw[->] (u) -> node[midway, above]{1} (v);
	\draw[->] (v) -> node[midway, above]{1} (w);
	\draw[->] (w) -> node[midway, above]{4} (t);
	% \foreach \i [evaluate = \i as \lastNode using \i-1] in {2,3,...,\numberOfNodes}
	% 	{
	% 		\node (Top\i) at (\i,1) [node, bright] {\i}
	% 		edge[<-] (Top\lastNode);
	% 	}


	% % Pfeile nach rechts
	% \pgfmathparse{\numberOfNodes - 2}
	% \foreach \i [evaluate = \i as \nextNode using \i+2] in {1,2,...,\pgfmathresult}
	% 	{
	% 		\foreach \j [count=\nodeIndex from \nextNode] in {\nextNode,...,\numberOfNodes}
	% 			{
	% 				\draw[->, lightEdgeStyle] (Bot\i) to [bend right] (Bot\nodeIndex);
	% 			}
	% 	}

\end{tikzpicture}

	\caption{An example graph where, depending on the driving time constraints, a long break at can render a short break obsolete.}
	\label{fig:graph_short_long_break}
\end{figure}

This is an optimistic assumption since not in all cases, a longer break renders a short break obsolete. Revisit the example graph of fig. \ref{fig:graph_short_long_break} with permitted driving times of $4$ and $5$. We still need one long and two short breaks, but the long break now must take place at $v$ while the short breaks must take place at $u$ and $w$. The long break did not spare a short break. Since we are searching for a lower bound for the amount of breaks, optimistic assumptions are necessary. Given a label $l \in L(v)$, we now can calculate a lower bound estimate for the amount of necessary breaks for each restriction $r_i$.

\begin{align}\label{eq:number_breaks_sum}
	\breakEstimate_i(l,v)  =\begin{dcases}
		\parbox[t]{.55\textwidth}{$\minBreaks_i(\breakDist_i(l) + \chpotential_t(v)) \\- \sum_{j=i+1}^{n}{\breakEstimate_j(l,v) }$} & 0 < i < n \\
		\minBreaks_n(\breakDist_n(l) + \chpotential_t(v)) & i=n
	\end{dcases}
\end{align}

Since we just subtract all break estimates for driving time constraints greater $i$ to obtain the estimate for $i$, we can just use

\begin{align}\label{eq:break_estimate_n}
	\breakEstimate_i(l,v) = \begin{dcases}
		\parbox[t]{.55\textwidth}{$\minBreaks_i(\breakDist_i(l) + \chpotential_t(v)) \\- \minBreaks_{i+1}(\breakDist_{i+1}(l) + \chpotential_t(v))$} & 0 < i < n \\
		\minBreaks_n(\breakDist_n(l) + \chpotential_t(v)) & i=n
	\end{dcases}
\end{align}

Finally, we can define the lower bound potential for a label $l \in L(v)$ and a target node $t$ as

\begin{align}\label{eq:multiple_breaks_pot}
	\begin{split}
		\concretepotential_t(l,v) & =\chpotential(v,t) + \sum_{i=1}^n{ \breakEstimate_i(l,v) \cdot \restr_{i,b}}\\
		& = \distance(v,t) + \sum_{i=1}^n{ \breakEstimate_i(l,v) \cdot \restr_{i,b}}
	\end{split}
\end{align}

We have to proof that queue keys still cannot decrease when propagating labels as as in lemma \ref{lemma:pot_labels_get_larger}. If that is true, lemma \ref{lemma:pot_lower_bound_csp} and theorem \ref{theorem:pot_stop_criterion} follow as a consequence.

\begin{lemma}
	Lemma \ref{lemma:pot_labels_get_larger} still holds for two driving time constraints.
\end{lemma}

\begin{proof}
	We follow the same outline as in the proof of lemma \ref{lemma:pot_lower_bound_csp} and therefore revisit the procedure \textsc{relaxEdge} at an edge $(u,v) \in E$ which now can produce three labels at $v$ for each label at $u$: not parking at $v$, short break at $v$, and long break at $v$.

	Following general observations can be made in an addition to the proof of lemma \ref{lemma:pot_labels_get_larger}:

	\begin{enumerate}
		\item[6.] Propagating a label $l \in L(u)$ to obtain a label $l' \in L(v)$ $\implies$ $\breakEstimate(l',v) \ge \breakEstimate(l,u)$\\(because of non-negative edge weights and there can only be more or longer breaks for greater distances)
	\end{enumerate}

	\emph{Case 1: Not parking at $v$}. In this case, $\concretett(l') = \concretett(l) + \len(u,v)$ and $\breakDist_i(l') = \breakDist_i(l) + \len(u,v)$.

	\begin{align}
		\begin{split}\label{eq:label_n_feasibility_proof_1}
			&\concretett(l') - \concretett(l) - \concretepotential_t(l,u) + \concretepotential_t(l',v)\\
			&= \concretett(l) + \len(u,v) - \concretett(l)\\
			& \phantom{{}=1} - (\breakEstimate_1(l,u) \cdot \restr_{1,b} + \breakEstimate_2(l,u) \cdot \restr_{2,b} + \chpotential(u))\\
			& \phantom{{}=1} + (\breakEstimate_1(l',v) \cdot \restr_{1,b} + \breakEstimate_2(l',v) \cdot \restr_{2,b} + \chpotential(v))\\
			&= \len(u,v) + \breakEstimate_1(l',v) \cdot \restr_{1,b} + \breakEstimate_2(l',v) \cdot \restr_{2,b} \\
			& \phantom{{}=1} - (\breakEstimate_1(l,u) \cdot \restr_{1,b} + \breakEstimate_2(l,u) \cdot \restr_{2,b}) - \chpotential(u) + \chpotential(v)\\
			&= \len(u,v) + \minBreaks_2(\breakDist_2(l') + \chpotential_t(v)) \cdot \restr_{2,b}\\
			& \phantom{{}=1} + (\minBreaks_1(\breakDist_1(l') + \chpotential_t(v))\\
			& \phantom{{}=1} - \minBreaks_{2}(\breakDist_{2}(l') + \chpotential_t(v))) \cdot \restr_{1,b} \\
			& \phantom{{}=1} - (\breakEstimate_1(l,u) \cdot \restr_{1,b} + \breakEstimate_2(l,u) \cdot \restr_{2,b}) - \chpotential(u) + \chpotential(v)\\
			&= \len(u,v) + \minBreaks_2(\breakDist_2(l) + \len(u,v) + \chpotential_t(v)) \cdot \restr_{2,b}\\
			& \phantom{{}=1} + (\minBreaks_1(\breakDist_1(l) + \len(u,v) + \chpotential_t(v))\\
			& \phantom{{}=1} - \minBreaks_{2}(\breakDist_{2}(l) + \len(u,v) + \chpotential_t(v))) \cdot \restr_{1,b} \\
			& \phantom{{}=1} - (\breakEstimate_1(l,u) \cdot \restr_{1,b} + \breakEstimate_2(l,u) \cdot \restr_{2,b}) - \chpotential(u) + \chpotential(v)\\
			&\stackrel{\text{(5. and 6.)}}{\ge} \len(u,v) + \minBreaks_2(\breakDist_2(l) + \chpotential_t(u)) \cdot \restr_{2,b}\\
			& \phantom{{}=1} + (\minBreaks_1(\breakDist_1(l) + \chpotential_t(u))\\
			& \phantom{{}=1} - \minBreaks_{2}(\breakDist_{2}(l) + \chpotential_t(u))) \cdot \restr_{1,b} \\
			& \phantom{{}=1} - (\breakEstimate_1(l,u) \cdot \restr_{1,b} + \breakEstimate_2(l,u) \cdot \restr_{2,b}) - \chpotential(u) + \chpotential(v)\\
			&= \len(u,v) + \breakEstimate_2(l,u) \cdot \restr_{2,b} + \breakEstimate_1(l,u) \cdot \restr_{1,b} \\
			& \phantom{{}=1} - (\breakEstimate_1(l,u) \cdot \restr_{1,b} + \breakEstimate_2(l,u) \cdot \restr_{2,b}) - \chpotential(u) + \chpotential(v)\\
			&= \len(u,v) - \chpotential(u) + \chpotential(v)\\
			&\stackrel{\text{(4.)}}{\ge} 0
		\end{split}
	\end{align}

	\emph{Case 2: Short break at $v$}. In this case, $\concretett(l') = \concretett(l) + \len(u,v) + \restr_b$ and $\breakDist_1(l') = 0$ and $\breakDist_2(l') = \breakDist_2(l) + \len(u,v)$.

	\begin{align}
		\begin{split}\label{eq:label_n_feasibility_proof_2}
			&\concretett(l') - \concretett(l) - \concretepotential_t(l,u) + \concretepotential_t(l',v)\\
			&= \concretett(l) + \len(u,v) + \restr_{1,b} - \concretett(l)\\
			& \phantom{{}=1} - (\breakEstimate_1(l,u) \cdot \restr_{1,b} + \breakEstimate_2(l,u) \cdot \restr_{2,b} + \chpotential(u))\\
			& \phantom{{}=1} + (\breakEstimate_1(l',v) \cdot \restr_{1,b} + \breakEstimate_2(l',v) \cdot \restr_{2,b} + \chpotential(v))\\
			&= \len(u,v) + \restr_{1,b} + \breakEstimate_1(l',v) \cdot \restr_{1,b} + \breakEstimate_2(l',v) \cdot \restr_{2,b} \\
			& \phantom{{}=1} - (\breakEstimate_1(l,u) \cdot \restr_{1,b} + \breakEstimate_2(l,u) \cdot \restr_{2,b}) - \chpotential(u) + \chpotential(v)\\
			&= \len(u,v) + \restr_{1,b} + \minBreaks_2(\breakDist_2(l) + \len(u,v) + \chpotential_t(v)) \cdot \restr_{2,b}\\
			& \phantom{{}=1} + (\minBreaks_1(0 + \chpotential_t(v))\\
			& \phantom{{}=1} - \minBreaks_{2}(\breakDist_{2}(l) + \len(u,v) + \chpotential_t(v))) \cdot \restr_{1,b} \\
			& \phantom{{}=1} - (\breakEstimate_1(l,u) \cdot \restr_{1,b} + \breakEstimate_2(l,u) \cdot \restr_{2,b}) - \chpotential(u) + \chpotential(v)\\
			&\stackrel{\text{(2.)}}{=} \len(u,v) + \minBreaks_2(\breakDist_2(l) + \len(u,v) + \chpotential_t(v)) \cdot \restr_{2,b}\\
			& \phantom{{}=1} + (\minBreaks_1(r_{1,d} + \chpotential_t(v))\\
			& \phantom{{}=1} - \minBreaks_{2}(\breakDist_{2}(l) + \len(u,v) + \chpotential_t(v))) \cdot \restr_{1,b} \\
			& \phantom{{}=1} - (\breakEstimate_1(l,u) \cdot \restr_{1,b} + \breakEstimate_2(l,u) \cdot \restr_{2,b}) - \chpotential(u) + \chpotential(v)\\
			&\stackrel{\text{(1. and 3.)}}{\ge} \len(u,v) + \minBreaks_2(\breakDist_2(l) + \len(u,v) + \chpotential_t(v)) \cdot \restr_{2,b}\\
			& \phantom{{}=1} + (\minBreaks_1(\breakDist_1(l) + \len(u,v) + \chpotential_t(v))\\
			& \phantom{{}=1} - \minBreaks_{2}(\breakDist_{2}(l) + \len(u,v) + \chpotential_t(v))) \cdot \restr_{1,b} \\
			& \phantom{{}=1} - (\breakEstimate_1(l,u) \cdot \restr_{1,b} + \breakEstimate_2(l,u) \cdot \restr_{2,b}) - \chpotential(u) + \chpotential(v)\\
			&\stackrel{\text{(5. and 4.)}}{\ge} \len(u,v) + \minBreaks_2(\breakDist_2(l) + \chpotential_t(u)) \cdot \restr_{2,b}\\
			& \phantom{{}=1} + (\minBreaks_1(\breakDist_1(l) + \chpotential_t(u))\\
			& \phantom{{}=1} - \minBreaks_{2}(\breakDist_{2}(l) + \chpotential_t(u))) \cdot \restr_{1,b} \\
			& \phantom{{}=1} - (\breakEstimate_1(l,u) \cdot \restr_{1,b} + \breakEstimate_2(l,u) \cdot \restr_{2,b}) - \chpotential(u) + \chpotential(v)\\
			&= \len(u,v) + \breakEstimate_2(l,u) \cdot \restr_{2,b} + \breakEstimate_1(l,u) \cdot \restr_{1,b} \\
			& \phantom{{}=1} - (\breakEstimate_1(l,u) \cdot \restr_{1,b} + \breakEstimate_2(l,u) \cdot \restr_{2,b}) - \chpotential(u) + \chpotential(v)\\
			&= \len(u,v) - \chpotential(u) + \chpotential(v)\\
			&\stackrel{\text{(4.)}}{\ge} 0
		\end{split}
	\end{align}


	\emph{Case 3: Long break at $v$}. In this case, $\concretett(l') = \concretett(l) + \len(u,v) + \restr_b$ and $\breakDist_i(l') = 0$.

	\begin{align}
		\begin{split}\label{eq:label_n_feasibility_proof_3}
			&\concretett(l') - \concretett(l) - \concretepotential_t(l,u) + \concretepotential_t(l',v)\\
			&= \concretett(l) + \len(u,v) + \restr_{2,b} - \concretett(l)\\
			& \phantom{{}=1} - (\breakEstimate_1(l,u) \cdot \restr_{2,b} + \breakEstimate_2(l,u) \cdot \restr_{2,b} + \chpotential(u))\\
			& \phantom{{}=1} + (\breakEstimate_1(l',v) \cdot \restr_{2,b} + \breakEstimate_2(l',v) \cdot \restr_{2,b} + \chpotential(v))\\
			&= \len(u,v) + \restr_{2,b} + \breakEstimate_1(l',v) \cdot \restr_{1,b} + \breakEstimate_2(l',v) \cdot \restr_{2,b} \\
			& \phantom{{}=1} - (\breakEstimate_1(l,u) \cdot \restr_{1,b} + \breakEstimate_2(l,u) \cdot \restr_{2,b}) - \chpotential(u) + \chpotential(v)\\
			&\stackrel{\text{(2.) and TODO is label}}{=} \len(u,v) + \minBreaks_2(\restr_{2,d} + \chpotential_t(v)) \cdot \restr_{2,b}\\
			& \phantom{{}=1} + (\minBreaks_1(\restr_{2,d} + \chpotential_t(v)) - \minBreaks_{2}(\restr_{2,d} + \chpotential_t(v))) \cdot \restr_{1,b} \\
			& \phantom{{}=1} - (\breakEstimate_1(l,u) \cdot \restr_{1,b} + \breakEstimate_2(l,u) \cdot \restr_{2,b}) - \chpotential(u) + \chpotential(v)\\
			&\stackrel{\text{(1. and 3. and $\restr_{2,d} >= \restr_{1,d}$)}}{\ge} \len(u,v) + \minBreaks_2(\breakDist_2(l) + \len(u,v) + \chpotential_t(v)) \cdot \restr_{2,b}\\
			& \phantom{{}=1} + (\minBreaks_1(\breakDist_1(l) + \len(u,v) + \chpotential_t(v))\\
			& \phantom{{}=1} - \minBreaks_{2}(\breakDist_2(l) + \len(u,v) +\chpotential_t(v))) \cdot \restr_{1,b} \\
			& \phantom{{}=1} - (\breakEstimate_1(l,u) \cdot \restr_{1,b} + \breakEstimate_2(l,u) \cdot \restr_{2,b}) - \chpotential(u) + \chpotential(v)\\
			&\stackrel{\text{(5. and 4.)}}{\ge} \len(u,v) + \minBreaks_2(\breakDist_2(l) + \chpotential_t(u)) \cdot \restr_{2,b}\\
			& \phantom{{}=1} + (\minBreaks_1(\breakDist_1(l) + \chpotential_t(u))\\
			& \phantom{{}=1} - \minBreaks_{2}(\breakDist_{2}(l) + \chpotential_t(u))) \cdot \restr_{1,b} \\
			& \phantom{{}=1} - (\breakEstimate_1(l,u) \cdot \restr_{1,b} + \breakEstimate_2(l,u) \cdot \restr_{2,b}) - \chpotential(u) + \chpotential(v)\\
			&= \len(u,v) + \breakEstimate_2(l,u) \cdot \restr_{2,b} + \breakEstimate_1(l,u) \cdot \restr_{1,b} \\
			& \phantom{{}=1} - (\breakEstimate_1(l,u) \cdot \restr_{1,b} + \breakEstimate_2(l,u) \cdot \restr_{2,b}) - \chpotential(u) + \chpotential(v)\\
			&= \len(u,v) - \chpotential(u) + \chpotential(v)\\
			&\stackrel{\text{(4.)}}{\ge} 0
		\end{split}
	\end{align}

\end{proof}

% \begin{lemma}
% 	On shortest $s$-$t$ paths, each long break at a node $v$ with a break time of $r_{2,b}$ also includes a necessary short break. In other words, you cannot remove a long break from a shortest path without also violating the short driving time constraint $r_1$.
% \end{lemma}

% \begin{proof}
% 	It is trivial that on a shortest path, it must not be possible to remove a long break from the path without violating $r_2$. We show that this is also true for $r_1$.

% 	Let $u$, $v$, and $w$ be nodes on a shortest $s$-$t$ path. Let there be a long break at $v$. Therefore, it must be $\drivingtime(s,v) \ge \restr_{2,d}$ or $\drivingtime(v,t) \ge \restr_{2,d}$, so w.l.o.g let $\drivingtime(s,v) \ge \restr_{2,d}$. Let's assume that we can remove the break at $v$ without violating $r_1$. Then there must exist a node $u$ with $\drivingtime(u,v) < \restr_{1,d}$.
% \end{proof}

\section{Bidirectional Goal-Directed Search with Multiple Driving Time Constraints}
We now extend the goal-directed approach of section \ref{section:potential_n_csp} to a bidirectional approach. Our algorithm will consist of a forward search from $s$ in $\overrightarrow{G}$ and a backward search from $t$ in $\overleftarrow{G}$. The distances of the forward and backward search are combined at nodes were which were settled by both searches. We therefore introduce a concept to merge the label sets of backward and forward search to find the best currently known and valid path using information of both searches. Since we aim to stop the search as early as possible, we have to decide on a stopping criterion which allows the search to stop way before the forward search settles the target node $t$ or the backward search settles $s$. The correctness of the stopping criterion is closely tied to the potential of section \ref{section:potential_n_csp}.

The input of the search remains a graph $G=(V,E,\mathfunction{len})$, a set of parking nodes $P \subseteq V$, a set of driving time constraints $\restrset$, and start and target nodes $s,t \in V$. There are two potentials $\overrightarrow{\concretepotential_t}$ and $\overleftarrow{\concretepotential}_s$ which we call the forward and the backward potential. The forward potential yields lower bounds for the remaining travel time of a label to $t$ in $\overrightarrow{G}=G$. The backward potential yields lower bounds for the remaining travel time of a label to $s$ in $\overleftarrow{G}$. The forward search then is a normal A* search on $\overrightarrow{G}$ with start node $s$ and target node $t$ and the backward search is a normal A* search on $\overleftarrow{G}$ with start node $t$ and target node $s$. Each search owns a queue of labels $\overrightarrow{Q}$ and $\overleftarrow{Q}$ and a label set $\overrightarrow{L}(v)$, respectively $\overleftarrow{L}(v)$ for each $v \in V$.

During the search, forward and backward search alternately settle nodes until the stopping criterion is met, one search completed the search by itself, or the queues ran empty. We hold the tentative value for $\traveltime(s,t)$ in a variable $\tenttraveltime(s,t)$ which we initialize with $\infty$ before settling the first node. When forward or backward search settles a node $v$, they additionally check if the label set of the other search at $v$ contains any settled labels. If this is the case, forward and backward search met at this node. We then search for the combination of labels which yields the shortest valid route between $s$ and $t$ via $v$. In other words, we want to find the labels $l \in \overrightarrow{L}(v)$ and $m \in \overleftarrow{L}(v)$ which minimize $\concretett(l) + \concretett(m)$ and for which $\breakDist_1(l) + \breakDist_1(m) < \restr_{1,d}$ and $\breakDist_1(l) + \breakDist_1(m) < \restr_{2,d}$. If the resulting distance for an $s$-$t$ path via $v$ is smaller than the previously known minimum tentative distance $\tenttraveltime(s,t)$, we update $\tenttraveltime(s,t)$ accordingly. We stop the forward search if the minimum key of $\overrightarrow{Q}$ is greater than $\tenttraveltime(s,t)$ and stop the backward search when the minimum key of $\overleftarrow{Q}$ is greater than $\tenttraveltime(s,t)$.

\begin{theorem}
	At the point in time when forward and backward search have stopped, $\tenttraveltime(s,t) = \traveltime(s,t)$. In other words, when the search stops, $\tenttraveltime(s,t)$ equals the minimum travel time from $s$ to $t$ which complies with the driving time constraints $\restrset$.
\end{theorem}

\begin{proof}
	We show that when the search stops, all valid $s$-$t$ routes $q$ which comply with the driving time constraints $\restrset$ and which were not found yet yield a larger travel time $\concretett(q)$ than the current $\tenttraveltime(s,t)$. This also implies that if $\tenttraveltime(s,t) = \infty$ at the point in time when the search stops, there exist no paths from $s$ to $t$.

	Since the search stops when the minimum keys of $\overrightarrow{Q}$ and $\overleftarrow{Q}$ are both greater than $\tenttraveltime(s,t)$, all labels $l$ which will be settled by continuing the search have a greater distance $\concretett(l)$. Therefore, if any new connection between forward and backward search which complies with the driving time constraints will be found, its distance will be greater than $\tenttraveltime(s,t)$.

	The shortest path with travel time $\traveltime(s,t)$ consists of two subpaths of forward and backward search which were connected at a node $v$. There exist label $l \in \overrightarrow{L}$ and $m \in \overleftarrow{L}$ with $\concretett(l) + \concretett(m) = \traveltime(s,t)$. Each label is the result of a unidirectional search from $s$, respectively from $t$. In section \ref{sec:dijkstra_csp_correctness} we proved that a unidirectional search can be stopped when the first label at its target node was removed from the queue. Since $l$ and $m$ are both smaller or equal to $\tenttraveltime(s,t)$ and both queue keys of forward and backward queue are greater, $l$ and $m$ where already removed from the respective queue. Therefore, we know that $\overrightarrow{L}(v)$ and $\overleftarrow{L}(v)$ contain the labels with the shortest distance from $s$ to $v$, respectively from $t$ to $v$. Consequently, when the second of both labels was settled at $v$, $\traveltime(s,t)$ was updated with the value $\concretett(l) + \concretett(m)$.
\end{proof}

\section{Core Contraction Hierarchy Search}
Given a graph $G = (V,E,\mathfunction{len})$, we construct a core contraction hierarchy in which the core contains all the parking nodes $P \subseteq V$. We denote the set of uncontracted core nodes as $C \subseteq V$. It is $P \subseteq C$. The set of nodes $V$ therefore is split into a set of core nodes $C$ and a set of contracted nodes $V_{CH} = V \setminus C$. The graph $G_{CH} = (V_{CH},E_{CH},\mathfunction{len})$ with nodes $V_{CH}$ and edges $E_{CH} = \{(u,v) \in E : u,v \in V_{CH}\}$ therefore is a valid contraction hierarchy. It contains only contracted nodes and edges between those nodes. Thus, $E_{CH}$ contains only upward edges. The graph $G_{C} = (V \setminus V_{CH},E \setminus E_{CH},\mathfunction{len})$ is called the core graph.

The core CH query essentially is a bidirectional Dijkstra with a modified stopping criterion. It operates on a modified forward graph $\overrightarrow{G}^*$ and a modified backward graph $\overleftarrow{G}^*$. The forward graph $\overrightarrow{G}^* = (V,\overrightarrow{E}^*,\mathfunction{len})$ consists of the forward graph of the contraction hierarchy $\overrightarrow{G}_{CH}$, extended by the core graph $G_C$. It is $\overrightarrow{E}^* = \overrightarrow{E}_{CH} \cup E_C $. Equivalently, the backward graph is defined as $\overleftarrow{G}^* = (V,\overleftarrow{E}^*,\overleftarrow{\mathfunction{len}})$ with $\overleftarrow{E}^* = \overleftarrow{E}_{CH} \cup E_C$ and $\overleftarrow{\mathfunction{len}}(u,v) = \mathfunction{len}(v,u)$.

The bidirectional query in $G^*$ consists of a forward search from $s$ in $\overrightarrow{G}^*$ and backward search from $t$ in $\overleftarrow{G}^*$. Each search consists of two parts, an upward search in $G_{CH}$ and a Dijkstra search in $G_C$. A search may begin at a core node and only perform the Dijkstra search. It also may not reach the core graph at all and only perform the CH upward search.

The stopping criterion of the bidirectional search must take into account that the graph $G^*$ contains the contraction hierarchy $G_{CH}$. The common stopping criterion for $s$-$t$ queries in a CH is to stop the search if the minimum queue key of both queues $\overrightarrow{Q}$ and $\overleftarrow{Q}$ is greater or equal to the tentative minimum distance $\distance(s,t)$ \cite{geisberger:2012}. Since this criterion is more conservative than the stopping criterion for a bidirectional Dijkstra search, we can use it for our combination of CH and bidirectional Dijkstra.

\begin{algorithm}[hbtp]
	\caption{\textsc{Core-CH with Driving Time Constraints}}\label{alg:CSPCoreCH}

	% Some settings
	\DontPrintSemicolon %dontprintsemicolon
	\SetFuncSty{textsc}
	\SetKwFor{ForAll}{forall}{do}

	% Declaration of data containers and functions
	\SetKwData{Q}{Q}
	\SetKwData{L}{L}
	\SetKwData{pot}{pot}
	\SetKwFunction{queueDeleteMin}{deleteMin}
	\SetKwFunction{queueInsert}{insert}
	\SetKwFunction{queueMin}{min}
	\SetKwFunction{queueMinKey}{minKey}
	\SetKwFunction{queueDecreaseKey}{decreaseKey}
	\SetKwFunction{queueContains}{contains}
	\SetKwFunction{listInsert}{insert}
	\SetKwFunction{removeDom}{removeDominated}

	% Algorithm interface
	\KwIn{Graph $G = (V,E,\mathfunction{len})$, parking nodes $P \subseteq V$, driving time constraint $\restr$, potential \pot{}, source node $s \in V$}
	\KwData{Priority queue \Q, per node priority queue \L{$v$} of labels for all $v \in V$}
	\KwOut{Distances for all $v \in V$, tree of allowed shortest paths according to the restriction $\restr$ from $s$, given by $l_{pred}$}

	% The algorithm
	\BlankLine
	\tcp{Initialization}
	\Q.\queueInsert{$s,(0,0)$}\;
	\L{$s$}.\queueInsert{$(\bot,\bot),\pot{(0,0)}$}\;
	$fwNext \leftarrow true$\;
	\BlankLine
	\tcp{Main loop}
	\While{stopping criterion is not met}
	{
		\If{$fwNext$}{}
		$u \leftarrow$ \Q.\queueDeleteMin{} \;
		$(\concretett, \breakDist_1) \leftarrow$ \L{$u$}.\queueMinKey{} \;
		$l \leftarrow$ \L{$u$}.\queueDeleteMin{} \;
		\BlankLine
		\If{\L{$u$} is not empty}
		{
			$k_{dist} \leftarrow$ \L{$u$}.\queueMinKey{} \;
			\Q.\queueInsert{$u$, $k_{dist}$} \;
		}

		\ForAll{ $(u,v) \in E$ }
		{
			\If{$\concretett + \mathfunction{len}(u,v) < \restr_d$}
			{
				$D \leftarrow \{(\concretett + \mathfunction{len}(u,v), \breakDist_1 + \mathfunction{len}(u,v))\}$

				\If{$v \in P$}
				{
					$D$.\listInsert{$(\concretett + \mathfunction{len}(u,v) + \restr_b, 0)$}
				}

				\ForAll{ $x \in D$ }
				{
					\If{$x$ is not dominated by any label in \L{$v$}}
					{
						\L{$v$}.\removeDom{$x$} \;
						\L{$v$}.\queueInsert{$(l,(u,v)), x$} \;
						\uIf{\Q.\queueContains{v}}
						{
							\Q.\queueDecreaseKey{$v, x$}
						}
						\Else
						{
							\Q.\queueInsert{$v, x$}
						}
					}
				}
			}
		}
	}
\end{algorithm}

To proof the correctness of the search, we will differentiate between the cases where neither of the two searches reaches the core and where at least one of the searches does reach the core. TODO

\subsection{Building the Core Contraction Hierarchy}

\section{Goal-Directed Core Contraction Hierarchy Search\label{sec:astar_corech}}
%% evaluation.tex
%%

%% ==============
\chapter{Improvements and Implementation}
In this chapter, we describe detailed modifications to the algorithm described in section \ref{sec:astar_corech} which enable further improvements of running time in specific cases and in practice.

\section{Label Queue}

\section{Bidirectional Backward Pruning}
With a bidirectional A* search, we can use the progress and the potential of the backward search to prune the forward search and vice versa. [TODO cite] A label $l$ at a node $v$ which was propagated along an edge $(u,v)$ can be discarded if we can proof that all paths using the label are longer or equal to the tentative travel time $\tenttraveltime(s,t)$. We know the travel time $\concretett(l)$ of the label $l$ at $v$ and need to find a lower bound for the remaining distance to $t$. We will describe the pruning of the forward search, the backward search can be pruned accordingly.

The backwards queue $\overleftarrow{Q}$ contains labels with a key of $\concretett(l) + \concretepotential_s(l,v)$ for label $l \in \overleftarrow{L}(v)$. Labels are removed from the queue with an increasing key. Therefore, we know that if a label  TODO continue

\section{Core Contraction Hierarchy Stopping Criteria}

\section{Constructing the Core Contraction Hierarchy}

% !TeX root = thesis.tex
%% evaluation.tex
%%

%% ==============
\chapter{Evaluation\label{ch:Evaluation}}
In this Section, we evaluate the running time and behavior of our algorithms of chapter~\ref{ch:Algorithm} in various experiments and describe the underlying data. Our machine runs openSUSE Leap 15.3, has \SI{128}{\giga\byte} (8x\SI{16}{\giga\byte}) of \SI{2133}{\mega\hertz} DDR4 RAM, and a 4-core Intel Xeon E5-1630v3 CPU which runs at \SI{3.7}{\giga\hertz}. The code is written in Rust and compiled with cargo 1.64.0-nightly using the release profile with lto~=~true and codegen-units~=~1.

\subparagraph{Data.} Our data is a road network of Europe from Open Street Map\footnote{\url{https://download.geofabrik.de/europe-latest.osm.pbf} of March 22, 2022} (OSM). We extract the routing graph from the OSM data using RoutingKit\footnote{\url{https://github.com/RoutingKit/RoutingKit}}, respectively a custom extension\footnote{\url{https://github.com/maxoe/RoutingKit}} of RoutingKit which is capable of extracting parking nodes accordingly and building the core CH as described in Section \ref{sec:build_corech}. The obtained routing graph has $81.5$ million nodes and $190$ million edges. If not stated otherwise, our set $P$ of parking nodes consists of 6800 nodes which were selected due to their OSM attributes. We will later conduct experiments with different choices for $P$.

\subparagraph{Methodology.} All experiments are run sequentially. We conduct experiments regarding the preprocessing time of the OSM data and the running time of queries on the extracted routing graph. We average preprocessing running times over \num{10} runs and running times of $s$-$t$ queries over \num{10000} queries with $s$ and $t$ independently chosen uniformly at random for each query. If not stated otherwise, we use $\restr_1$ with $\restr_1^d = \SI{4.5}{\hour}$ and $\restr_1^b = \SI{0.45}{\minute}$ and $\restr_2$ with $\restr_2^d = \SI{9}{\hour}$ and $\restr_2^b = \SI{9}{\hour}$ to approximate the regulations of the EU.

\subsection{Comparison of Algorithms and Optimizations}
We evaluate the different algorithms of chapter \ref{ch:Algorithm} and the optimizations of chapter \refeq{ch:impl}.

First, we compare the running times of queries of the algorithms of chapter \ref{ch:Algorithm} on a German road network which was obtained in the exact same way from OSM as the European road network\footnote{\url{https://download.geofabrik.de/europe/germany-latest.osm.pbf} of March 22, 2022}. We scale the experiment down from the European road network because some variants of the algorithms in this experiment cannot keep up with the performance of the goal-directed core CH algorithm and would render the experiment slow and impracticable. To avoid distortions, we also scale down the driving time constraints. Otherwise, most random $s$-$t$ queries on the German road network would be to short to require a break and almost no query would require more than one break. We aim for a similar average number of breaks and average break time on a route. Therefore, we first adjust the maximum allowed driving time without a break and then the break time of a break.


The average travel time of all queries of Table \ref{tbl:extensions_runtime} is \SI{4}{\hour} \SI{43}{\second}, the average break time on a route is \SI{17.5}{\minute}. This leads to an average driving time of \SI{3}{\hour} \SI{43}{\minute} \SI{13}{\second}.

As Table \ref{tbl:extensions_runtime} shows, the goal-directed search already performs an order of magnitude better than the baseline Dijkstra's algorithm with our amendments for driving time constraints. TODO bidir. The largest improvement brings the bidirectional search which improves the baseline by a factor of about \num{10000} and the goal-directed variant by a factor of about \num{1000}.

\begin{table}[hbtp]
	\centering
	\begin{tabular}{cccrrrrrr}
	\toprule
	              &               &         & \multicolumn{2}{c}{Germany [\si{\milli\second}]} & \multicolumn{2}{c}{Europe [\si{\milli\second}]}                 \\
	Goal-Directed & Bidirectional & Core CH & 1-DTC                                            & 2-DTC                                           & 1-DTC & 2-DTC \\
	\midrule
	\xmark        & \xmark        & \xmark  & 35110.14                                                & 44083.92                                               & -     & -     \\
	\cmark        & \xmark        & \xmark  & 1393.72                                                & 7.87                                               & 15055.56     & -     \\
	\xmark        & \cmark        & \xmark  & 50463.27                                                & -                                               & -     & -     \\
	\cmark        & \cmark        & \xmark  & 4.38                                                & 14.56                                               & 30372.80     & -     \\
	\xmark        & \cmark        & \cmark  & 121.29                                                & 182.92                                               & 547.64     & 720.72     \\
	\cmark        & \cmark        & \cmark  & 3.47                                                & 4.15                                               & 126.06     & 339.97     \\
	\bottomrule
\end{tabular}
	\caption{Running times of random queries on a German road network}
	\label{tbl:extensions_runtime}
\end{table}

Most of the performance gain of the goal-directed search originates from the very tight lower-bound given by the CH potentials. If the shortest travel time between two nodes $s$ and $t$ is way larger than $\distance(s,t)$ due to necessary breaks and even detours to parking nodes on the route, then the performance of the goal-directed search degrades. The bidirectional variant can mitigate this disadvantage since it connects two routes which each for itself have fewer breaks on the route. A disadvantage from the goal-directed search which cannot be mitigated by the bidirectional variant are its outliers. In cases where the algorithm is not able to find a route or the route needs a lot of breaks, the running time increases significantly. TODO

TODO outlier problem with graphics leas to non ch excluded even if fast
\begin{table}[hbtp]
	\centering
	\begin{tabular}{cccrrrr}
	\toprule
	              &               &         & \multicolumn{2}{c}{Running Time [\si{\milli\second}]}         \\
	Goal-Directed & Bidirectional & Core CH & 1-DTC                                                 & 2-DTC \\
	\midrule
	\xmark        & \cmark        & \cmark  & 327.12                                                     & 327.12     \\
	\cmark        & \cmark        & \cmark  & 145.70                                                     & 145.70     \\
	\bottomrule
\end{tabular}
	\caption{Running times of random queries on a European road network}
	\label{tbl:extensions_runtime_eur}
\end{table}

Second, we evaluate the different optimizations as described in chapter \refeq{ch:impl} in use with the goal-directed core CH algorithm on the European road network.

\begin{table}[hbtp]
	\centering
	\begin{tabular}{ccrr}
	\toprule
	                 &                                      & \multicolumn{2}{c}{Running Time [\si{\milli\second}]}         \\
	Backward Pruning & Additional Core CH Stopping Criteria & 1-DTC                                                 & 2-DTC \\
	\midrule
	\xmark           & \xmark                               & @                                                     & @     \\
	\cmark           & \xmark                               & @                                                     & @     \\
	\xmark           & \cmark                               & @                                                     & @     \\
	\cmark           & \cmark                               & @                                                     & @     \\
	\bottomrule
\end{tabular}
	\caption{Comparison of running times of the goal-directed core CH algorithm with different optimizations from Section\ref{ch:impl}.}
	\label{tbl:opt_runtime}
\end{table}

\subsection{Study of Goal-Directed Core CH Queries}
Finally, we investigate queries of the goal-directed core CH algorithm with full optimizations more closely. What drives the running time of the algorithm? TODO

\begin{figure}[hbtp]
	\centering
	\includegraphics[width=.95\textwidth]{plots/measure_all_csp_2_1000_queries_rank_times-core_ch_chpot-time_ms.png}
	\caption{Running times of the goal-directed core CH algorithm for queries to nodes of increasing Dijkstra rank, logarithmic scales.}
	\label{fig:rank_times}
\end{figure}

\begin{itemize}
	\item different $\len$
	\item variable break time, driving time limit
	\item parking node set
\end{itemize}

%% conclusion.tex
%%

%% ==================
\chapter{Conclusion}
\label{ch:conclusion}
%% ==================
conclusion
- motivtion recap: long-haul
- abstraction tdrp
- algorithm baseline and goal-directed and core ch extensions
- evaluation shows ... results


outlook
- theoretical analysis of n dtc
- exact model of dtc
- merging with time-dependent for road closures and driving bans





%% --------------------
%% |   Bibliography   |
%% --------------------

\cleardoublepage
\phantomsection
\addcontentsline{toc}{chapter}{\bibname}

\iflanguage{english}
{\bibliographystyle{alpha}}
{\bibliographystyle{babalpha-fl}} % german style

% \bibliography{references,references_zotero,references_custom}
\bibliography{references_zotero,references_custom}



%% ----------------
%% |   Appendix   |
%% ----------------

\cleardoublepage
% !TeX root = thesis.tex
%% appendix.tex
%%

%% ==============================
%\chapter{Appendix}
%\label{ch:Appendix}
%% ==============================

\appendix

\iflanguage{english}
{\addchap{Appendix}}	% english style
{\addchap{Anhang}}	% german style

\section{List of Abbreviations}
\label{app:abb}

\begin{tabular}{rp{0.85\textwidth}}
	SPP           & Shortest Path Problem                                                                                                        \\
	TDSP          & Truck Driver Scheduling Problem                                                                                              \\
	MD-TDSP       & Minimum Duration Truck Driver Scheduling Problem                                                                             \\
	TDSRP         & Truck Driver Scheduling and Routing Problem                                                                                  \\
	TD-TDSRP      & Time-Dependent Truck Driver Scheduling and Routing Problem                                                                   \\
	TD-TDSRP-2B   & Time-Dependent Truck Driver Scheduling and Routing Problem with two types of breaks (two types of driving time restrictions) \\
	LH-TDRP       & Long-Haul Truck Driver Routing Problem                                                                                       \\
	OSM           & Open Street Map                                                                                                              \\
	1-DTC (2-DTC) & Restriction of the long-haul truck driver routing problem to only one (two) driving time constraint(s)                       \\
	CH            & Contraction Hierarchy                                                                                                        \\
	IQR           & Interquartile Range
\end{tabular}



\section{List of Symbols and Designations}
\label{app:symbols}

We use Greek symbols for theoretical and abstract concepts such as the shortest distance between two nodes or an arbitrary valid node potential. Concrete functions and procedures, i.e., those with definitions, have verbose names. Algorithmic functions and procedures for which pseudocode is provided use \algofunction{SmallCaps}.

\begin{tabular}{rp{0.8\textwidth}}
	$\potential_t(v)$         & A node potential to $t$                                                                                                                   \\
	$\distance(s,t)$          & Shortest distance between $s$ and $t$                                                                                                     \\
	$\traveltime(s,t)$        & Shortest travel time between $s$ and $t$                                                                                                  \\
	\\
	$\route$                  & A route consisting of a path and a function $\breakTime(v)$                                                                               \\
	$\restr$                  & A driving time constraint                                                                                                                 \\
	$c^d$                     & Maximum allowed driving time of a driving time constraint $c$                                                                             \\
	$c^b$                     & Minimum pause time of a driving time constraint $\restr$                                                                                  \\
	$\restrset$               & A set of one or more driving time constraints                                                                                             \\
	$L(v)$                    & The label set of a node $v$                                                                                                               \\
	$\breakDist_i(l)$         & Distance since the last break of a label with a duration of at least $c_i^b$ or distance since the last break if the index $i$ is omitted \\
	$\pred(l)$                & Predecessor label of a label $l$                                                                                                          \\

	$\concretett(l)$          & Travel time of a label $l$                                                                                                                \\
	$\concretepotential(l,v)$ & The potential of a label $l \in L(v)$ as defined in Section \ref{section:potential_n_csp}                                                 \\
	$\len((u,v))$             & The weight, respectively length of an edge $(u,v)$, alternatively used for the length $\len(p)$ of a path $p$                             \\
	$\ch$                     & A contraction hierarchy with vertices $\chv$ and edges $\che$                                                                             \\
	$\corech$                 & A core contraction hierarchy with vertices $\corechv$ and edges $\coreche$                                                                \\
\end{tabular}






\end{document}
