%% introduction.tex

\chapter{Introduction\label{ch:introduction}}

For many professional truck drivers, fatigue is a daily companion while en-route. Early surveys \cite{williamson:2001, adams-guppy:2003} interviewing drivers in different countries all over the world conclude that fatigue is experienced by drivers daily and within hours of the start of the drive. Supporting drivers in scheduling breaks as required can reduce fatigue. This is especially important since fatigue is a common cause for accidents and near misses. The investigation of \cite{evers:2005} observed severe truck accidents in Germany for a period of three months. They found that in $16\%$ of the 127 registered accidents that were caused by the truck driver, fatigue was mentioned as the cause of the accident. A study of causes of truck crashes in the US \cite{federalmotorcarriersafetyadministrationfmcsa:2006} examines 967 crashes during the years 2001 to 2003 and found that $13\%$ of the truck drivers stated that they experienced fatigue during the time of the crash. In \cite{connor:2001}, the authors find a strong relationship between acute fatigue and crash involvement by interviewing crash-involved and non-crash-involved drivers in the same areas and during the same times. In a similar case-control-study, \cite{cummings:2001} find a fourteen-fold increased crash risk for drivers who have experienced fatigue to a level where they almost have fallen asleep in the driver's seat. Today, fatigue is still a common cause for general accidents on the road, not only for accidents which involve trucks \cite{statistischesbundesamtdestatis:2021}, a finding that is also supported by a large meta-study \cite{moradi:2019} which additionally finds that strategies to reduce driver fatigue can effectively reduce the risk of traffic accidents.

Regulators reacted early to reduce the risk of accidents caused by fatigued truck drivers and introduced regulations, often using the term \emph{drivers' working hours} or \emph{hours of service} in the United States. In 1985, the Council of the European Communities (since 1993 the Council of the European Union) passed council regulation (EEC) No 3820/85 \cite{counciloftheeuropeancommunities:1985} which, with some exceptions, introduced a mandatory break of \SI{45}{\minute} after \SI{4.5}{\hour} of driving and a daily rest period of \SI{11}{\hour} for professional drivers ``involved in the carriage of goods''. In addition, drivers must extend the daily rest period to \si{45} consecutive hours once every week or, if exceptions apply, must compensate for a reduced weekly rest period within three weeks. Directive 2002/15/EC \cite{europeanparliament:2002} of the European Parliament and of the Council governs working times and introduced a maximum average weekly working time of \SI{48}{\hour} over four months and a maximum weekly working time of \SI{60}{\hour}. Since driving time is working time, the directive also applies for truck drivers. The regulation that determines the driving time limits, break, and rest times of today was introduced in 2006 as EC 561/2006 of the European Union \cite{europeanparliament:2006}. It contains core principles such as the break after \SI{4.5}{\hour} and the daily and weekly rest periods. Since this regulation is of greater importance of this work, we present a brief study of it in Section~\ref{sec:dwh_eu}.

In the US, the first hours of services regulations for long-haul truck drivers were adopted in the late 1930s - a time in which trucks achieved average speeds on routes of about \SI[per-mode = symbol]{40}{\km\per\hour} \cite{federalmotorcarriersafetyadministrationfmcsa:2000}. The regulation limited working hours to \SI{12}{\hour} within a period of \SI{15}{\hour} and introduced a weekly maximum of \SI{60}{\hour}. The US HOS regulations for truck drivers of today allow a maximum of \SI{8}{\hour} of driving until the driver must take a break for at least \SI{30}{\minute} and a maximum \SI{11}{\hour} of on-duty work until the driver must rest for \SI{10}{\hour} off duty \cite{federalmotorcarriersafetyadministrationfmcsa:2011}. As for the current EU regulation, we will present a more detailed study of the US rules in Section~\ref{sec:hos_us}.

Driver's working hours regulations increase the road safety and the working conditions of truck drivers. At the same, time they impose a burden on the truck driver who is responsible for following the regulations and taking the necessary breaks and rest times. Additionally, the regulations impose the challenge for the driver of finding appropriate places for parking along the route which cause as little as possible detours from the original route. To reduce the necessary manual routing effort from the truck driver, the search for parking locations for mandatory breaks ideally is incorporated into the navigation system which is used to navigate towards the destination. This necessitates routing algorithms which are capable of determining the need for breaks on a route and finding parking locations which lead to the smallest possible detours while complying with given drivers' working hours regulations.

Existing work manages to find the routes to a destination which minimize the sum of driving time and break time if the number of breaks on the route is limited to one break. The EU regulations allow routes up to a driving time of \SI{9}{\hour} with one break of \SI{45}{\minute} at the half-way point. Therefore, limiting the number of possible breaks on a route to one break is suitable for finding shorter routes with a travel time of not longer than a day, whereas long-haul truck drivers who drive multi-day routes are left out with this approach since the maximum travel time of a route which can be found with this limitation is limited. Additionally, long-haul truck drivers must consider a more complex set of regulations that mandates daily and weekly rest times, many of which are not relevant for planning short routes. We name the problem of finding a shortest route to a destination in an arbitrary distance while following a given set of driver's working hours regulations the Long-Haul Truck Driver Routing Problem (LH-TDRP). To the best of our knowledge, there exists no approach which allows an unlimited number of breaks on the route and achieves practicable performance.

In this thesis, we will begin with an overview over related literature on truck driver routing or scheduling in Chapter~\ref{ch:related_work} where we also discuss the driver's working hours regulations of the EU and the US. Chapter~\ref{ch:preliminaries} introduces our notations and the foundations on which we base our work. In Chapter~\ref{ch:problem_definitions}, we then abstract from the LH-TDRP and its variants using abstractions from the EU and US regulations to obtain the Truck Driver Routing Problem (TDRP), which we will define formally. In Chapter~\ref{ch:Algorithm}, we present a baseline algorithm and use the foundations of Chapter~\ref{ch:preliminaries} to develop extensions and optimizations that lead to practicable running times. Finally, we conduct a series of experiments regarding the running time of our algorithms on a German and European road network in Chapter~\ref{ch:Evaluation}. Additionally, we investigate how different parameters and road networks influences the performance of our algorithms. We conclude this work in Chapter \ref{ch:conclusion} with a short summary and suggestions on possible future research.